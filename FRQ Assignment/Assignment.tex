\documentclass{article}

\usepackage[a4paper, margin=1in]{geometry}
\usepackage{amsmath}
\usepackage{graphicx}

\setlength{\parskip}{1.5em}
\newcommand{\C}{\mathal{C}}
\newcommand{\qspace}{\vspace*{1.7em}}

\title{
    \textbf{AP Calculus AB: \\Free-Response Question}
    \author{Alexandru Stan}
    \date{March 10, 2024}
}

\begin{document}

    \maketitle
    \vfill
    \newpage

    \section*{Introduction}

    A Calculator is not allowed for this question, and, all AP guidlines (as stated below) 
    are to be followed. 

    Show all of your work, even though the question may not explicitly remind you to do so. Clearly label any
    functions, graphs, tables, or other objects that you use. Justifications require that you give mathematical reasons,
    and that you verify the needed conditions under which relevant theorems, properties, definitions, or tests are
    applied. Your work will be scored on the correctness and completeness of your methods as well as your answers.
    Answers without supporting work will usually not receive credit.

    Unless otherwise specified, answers (numeric or algebraic) need not be simplified. If your answer is given as a
    decimal approximation, it should be correct to three places after the decimal point.

    Unless otherwise specified, the domain of a function $f$ is assumed to be the set of all real numbers $x$ for which
    $f(x)$ is areal number.

    \section*{Question}

    \begin{table}[h]
        \def\arraystretch{2}
        \begin{tabular}{|c|c|c|c|c|c|}
            \hline
            $t_{\text{(hours)}}$ & 1 & 3 & 5 & 6 & 9 \\
            \hline
            $S(t)_{\text{(sales per hour)}}$ & 2 & 11 & 8 & 5 & 2 \\
            \hline 
        \end{tabular}
    \end{table} 

    The rate at which cars are sold at a dealership is modeled by $S(t)$ such that $S \in C^2$ 
    where  $S$ is measured in sales per hour and $t$ is measured in hours $0 \le t \le 11$ such that
    $t = 0$ is 8am. Values are given in the above table for selected values of $t$.

    \begin{enumerate}
        
        \item[(a)] Use the data in the table above to approximate $S\prime(4)$. Show the computations that lead to your answer
                   and indicate units of measure. \textbf{[3 pts]} \qspace 
        \item[(b)] Use a right Riemann sum with the four subinvtervals indicated by the data in the table to 
                   approximate $\int_1^3 R(t)dt$. Indicate units of measure. \textbf{[2 pts]} \qspace
        \item[(c)] Is the approximation in part (b) an overestimate or underestimate $\int_1^3$? \textbf{[2 pts]} \qspace
        \item[(d)] The sum $\sum_{k = 1}^n S\left(\frac{2}{n}(k - 1)\right)\frac{2}{n}$ is a left Riemann sum with $n$ subintervals of equal length. 
                   The limit of this sum as $n$ goes to $\infty$ can be stated as a definite integrals. Express the limit as a definite integral. \textbf{[2 pts]}

    \end{enumerate}

\end{document}