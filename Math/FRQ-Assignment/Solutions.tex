\documentclass[11pt]{article}
\usepackage{amsmath,amssymb,amsthm,enumerate,nicefrac,fancyhdr,hyperref,graphicx,adjustbox}
\hypersetup{colorlinks=true,urlcolor=blue,citecolor=blue,linkcolor=blue}
\usepackage[left=2.6cm, right=2.6cm, top=1.5cm, includehead, includefoot]{geometry}
\usepackage[dvipsnames]{xcolor}
\usepackage[d]{esvect}


%% header
\pagestyle{fancy}
\fancyhead[L]{\bf\large IDC4UP: AP Calculus AB \\ FRQ Assignment Solutions}
\fancyhead[R]{\bf\large March 2024 \\}
%\fancyfoot[C]{Page \thepage\ of 2}
\setlength{\headheight}{35pt}

\begin{document}

    \begin{enumerate}

		% (a)
		\item[(a)]
		
            \textbf{[1 pt]} To receive the first point, the student must understand that the Mean Value Theorem
            is required. They must also state it out in a fashion that is similar to the following: 

            \begin{center}
                \textit{The Mean Value Theorem states that if a function is continuous on the closed interval $[a, b]$ 
                and and differentiable on the open interval $(a, b)$, then there exists a number $c$ in the 
                open interval such that $f^{\prime}(c) = \frac{f(b) - f(a)}{b - a}$.}
            \end{center}

            \textbf{[1 pt]} Another point is then awarded if the student is able to recognize the correct values to plug in the correct values,
            to solve for $S^{\prime}(4)$, which, in this case, are $a = 3$ and $b = 5$.

            \textbf{[0.5pt]} The student is awarded a half point for returning the correct answer as follows below

            \[
                \begin{aligned}
                    S^{\prime}(4) &= \frac{f(5) - f(3)}{5 - 3} \\
                    S^{\prime}(4) &= \frac{8 - 11}{2} \\
                    S^{\prime}(4) &= -\frac{3}{2}
                \end{aligned}    
            \]

            \textbf{[0.5pt]} Lastly, another half-point is awarded for stating the correct units of measurement, 
            in this case, \textit{sales per hour per hour} or $\frac{\text{sales}}{h^2}$
		
		% (b)
		\item[(b)]
		
        \item[(b)]
        
        \item[(b)]
		
    \end{enumerate}
	
\end{document}