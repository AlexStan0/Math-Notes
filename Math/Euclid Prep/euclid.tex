\documentclass[11pt]{article}
\usepackage{amsmath,amssymb,amsthm,enumerate,nicefrac,fancyhdr,hyperref,graphicx,adjustbox}
\hypersetup{colorlinks=true,urlcolor=blue,citecolor=blue,linkcolor=blue}
\usepackage[left=2.6cm, right=2.6cm, top=1.5cm, includehead, includefoot]{geometry}
\usepackage[dvipsnames]{xcolor}
\usepackage[d]{esvect}

%% commands
%% useful macros [add to them as needed]
% sets
\newcommand{\C}{{\mathbb{C}}} 
\newcommand{\N}{{\mathbb{N}}}
\newcommand{\Q}{{\mathbb{Q}}}
\newcommand{\R}{{\mathbb{R}}}
\newcommand{\Z}{{\mathbb{Z}}}
\newcommand{\F}{{\mathbb{F}}}

% bases
\newcommand{\mA}{\mathcal{A}}
\newcommand{\mB}{\mathcal{B}}
\newcommand{\mC}{\mathcal{C}}
\newcommand{\mD}{\mathcal{D}}
\newcommand{\mE}{\mathcal{E}}
\newcommand{\mL}{\mathcal{L}}
\newcommand{\mM}{\mathcal{M}}
\newcommand{\mO}{\mathcal{O}}
\newcommand{\mP}{\mathcal{P}}
\newcommand{\mS}{\mathcal{S}}
\newcommand{\mT}{\mathcal{T}}

% linear algebra
\newcommand{\diag}{\operatorname{diag}}
\newcommand{\adj}{\operatorname{adj}}
\newcommand{\rank}{\operatorname{rank}}
\newcommand{\spn}{\operatorname{Span}}
\newcommand{\proj}{\operatorname{proj}}
\newcommand{\prp}{\operatorname{perp}}
\newcommand{\refl}{\operatorname{refl}}
\newcommand{\tr}{\operatorname{tr}}
\newcommand{\nul}{\operatorname{Null}}
\newcommand{\nully}{\operatorname{nullity}}
\newcommand{\range}{\operatorname{Range}}
\renewcommand{\ker}{\operatorname{Ker}}
\newcommand{\col}{\operatorname{Col}}
\newcommand{\row}{\operatorname{Row}}
\newcommand{\cof}{\operatorname{cof}}
\newcommand{\Num}{\operatorname{Num}}
\newcommand{\Id}{\operatorname{Id}}
\newcommand{\ipb}{\langle \thinspace, \rangle}
\newcommand{\ip}[2]{\left\langle #1, #2\right\rangle} % inner products
\newcommand{\M}[2]{M_{#1\times #2}(\F)}
\newcommand{\RREF}{\operatorname{RREF}}
\newcommand{\cv}[1]{\begin{bmatrix} #1 \end{bmatrix}}
\newenvironment{amatrix}[1]{\left[\begin{array}{@{}*{\numexpr#1-1}{c}|c@{}}}{\end{array}\right]}
\newcommand{\am}[2]{\begin{amatrix}{#1} #2 \end{amatrix}}
\newcommand{\lt}{\ensuremath <}

% vectors
\newcommand{\vzero}{\vv{0}}
\newcommand{\va}{\vv{a}}
\newcommand{\vb}{\vv{b}}
\newcommand{\vc}{\vv{c}}
\newcommand{\vd}{\vv{d}}
\newcommand{\ve}{\vv{e}}
\newcommand{\vf}{\vv{f}}
\newcommand{\vg}{\vv{g}}
\newcommand{\vh}{\vv{h}}
\newcommand{\vl}{\vv{\ell}}
\newcommand{\vm}{\vv{m}}
\newcommand{\vn}{\vv{n}}
\newcommand{\vp}{\vv{p}}
\newcommand{\vq}{\vv{q}}
\newcommand{\vr}{\vv{r}}
\newcommand{\vs}{\vv{s}}
\newcommand{\vt}{\vv{t}}
\newcommand{\vu}{\vv{u}}
\newcommand{\vvv}{{\vv{v}}}
\newcommand{\vw}{\vv{w}}
\newcommand{\vx}{\vv{x}}
\newcommand{\vy}{\vv{y}}
\newcommand{\vz}{\vv{z}}

% display
\newcommand{\ds}{\displaystyle}
\newcommand{\qand}{\quad\text{and}}
\newcommand{\qandq}{\quad\text{and}\quad}
\newcommand{\hint}{\textbf{Hint: }}

% misc
\newcommand{\area}{\operatorname{area}}
\newcommand{\vol}{\operatorname{vol}}
\newcommand{\red}[1]{{\color{red} #1}}
\newcommand{\rc}{\red{\checkmark}}

%% header
\pagestyle{fancy}
\fancyhead[L]{\bf\large MATH136: Linear Algebra 1 \\ Written Assignment 1 Solutions}
\fancyhead[R]{\bf\large Winter 2023 \\}
%\fancyfoot[C]{Page \thepage\ of 2}
\setlength{\headheight}{35pt}

\begin{document}
	\begin{enumerate}[{\bf Q1.}] 
	
	%Q1
	\item
  	\begin{enumerate}
		%Q1(a)
		\item 

        \[
            \begin{aligned}
                \frac{3^2 - 2^3}{2^3 - 3^2} &= \frac{9 - 8}{8 - 9} \\
                &= \frac{1}{-1} \\
                &= -1
            \end{aligned}    
        \]

		%Q1(b)
		\item 

        \[
            \begin{aligned}
                \sqrt{\sqrt{81} + \sqrt{9} - \sqrt{64}} &= \sqrt{9 + 3 - 8}\\
                &= \sqrt{4} \\
                &= 2
            \end{aligned}    
        \]

		\item 

        \[
            \begin{aligned}
                \frac{1}{\sqrt{x^2 + 7}} &= \frac{1}{4}\\
                \sqrt{x^2 + 7} &= 4\\
                x^2 + 7 &= 16\\
                x^2 &= 9\\
                x &= \pm 3
            \end{aligned}
        \]
		
	\end{enumerate}
	
	
	% New page for Crowdmark
	\newpage
	
    \item
  	\begin{enumerate}
		%Q1(a)
		\item $1 \lt a \lt b$ and $ab = 2022$

        \begin{itemize}
            \item $(2, 1011)$ 
            \item $(3, 674)$
            \item $(6, 337)$
        \end{itemize}

		%Q1(b)
		\item 

         \[
            \begin{aligned}
                \frac{2c + 1}{2d + 1} &= \frac{1}{17} \\
                2d + 1 &= 17(2c + 1)\\
                2d + 1 &= 34c + 17\\
                2d = 34c + 16\\
                d = 17c + 8\\
            \end{aligned}   
         \]

         \[
            \begin{aligned}
                d &> 0\\
                17c + 8 &> 0\\
                17c &> -8\\
                c &> \frac{-8}{17}
            \end{aligned}   
         \]

         \[
            \begin{aligned}
                d &= 17c + 8\\
                17c &= d - 8\\
                c &= \frac{d-8}{17}
            \end{aligned}
         \]

         \[
            \begin{aligned}
                c &> 0\\
                \frac{d-8}{17} &> 0\\
                d - 8 &> 0\\
                d &> 8\\
            \end{aligned}   
         \]

         $\therefore$ the lowest value $d$ can be is $8$

		\item $(px + r)(x + 5) = x^2 + 3x + t$
		
        As $a = 1$ then $p = 1$

        let $t = 5r$, as $b = 3$ then $5 + r = 3$

        \[
            \begin{aligned}
                5 + r &= 3\\
                r &= -2
            \end{aligned}    
        \]

        As $t = 5r$ then $t = 5(-2)$ or $t = -10$
		
	\end{enumerate}

    \newpage

    \item
  	\begin{enumerate}
		%Q1(a)
		\item

		%Q1(b)
		\item 

		\item 
		
	\end{enumerate}

    \newpage

    \item
  	\begin{enumerate}
		%Q1(a)
		\item

		%Q1(b)
		\item 

		\item 
		
	\end{enumerate}

    \newpage

    \item
  	\begin{enumerate}
		%Q1(a)
		\item

		%Q1(b)
		\item 

		\item 
		
	\end{enumerate}

    \newpage

    \item
  	\begin{enumerate}
		%Q1(a)
		\item

		%Q1(b)
		\item 

		\item 
		
	\end{enumerate}

    \newpage

    \item
  	\begin{enumerate}
		%Q1(a)
		\item

		%Q1(b)
		\item 

		\item 
		
	\end{enumerate}

    \newpage

    \item
  	\begin{enumerate}
		%Q1(a)
		\item

		%Q1(b)
		\item 

		\item 
		
	\end{enumerate}

    \newpage

    \item
  	\begin{enumerate}
		%Q1(a)
		\item

		%Q1(b)
		\item 

		\item 
		
	\end{enumerate}

    \newpage

    \item
  	\begin{enumerate}
		%Q1(a)
		\item

		%Q1(b)
		\item 

		\item 
		
	\end{enumerate}

    \newpage
	
\end{enumerate}
\end{document}