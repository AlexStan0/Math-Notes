\documentclass[12pt]{article}

% Import dependencies
\usepackage[utf8]{inputenc}
\usepackage[english]{babel}
\usepackage{indentfirst}
\usepackage{graphicx}
\usepackage{caption}
\usepackage{float}
\usepackage[margin=1.2in]{geometry} % margins
\usepackage{multicol}
\usepackage{wrapfig}
\usepackage{amsmath}
\usepackage{amssymb}
\usepackage[colorlinks=true, linkcolor=blue, urlcolor=cyan]{hyperref}
\usepackage{fancyhdr}
\usepackage{tikz}

\pagestyle{fancy}

\fancyhead[L]{\bf\large AP Calc - Complex Numbers}
\fancyhead[R]{\bf\large Sem. 2}
\setlength{\headheight}{35pt}

\newcommand{\C}{{\mathbb{C}}} 
\newcommand{\N}{{\mathbb{N}}}
\newcommand{\Q}{{\mathbb{Q}}}
\newcommand{\R}{{\mathbb{R}}}
\newcommand{\Z}{{\mathbb{Z}}}
\newcommand{\F}{{\mathbb{F}}}

\DeclareMathOperator{\arcsec}{arcsec}
\DeclareMathOperator{\arccot}{arccot}
\DeclareMathOperator{\arccsc}{arccsc}

\renewcommand{\baselinestretch}{1.5}
\newcommand{\fline}{\par\noindent\rule{\textwidth}{0.1pt}}
\newcommand{\uit}[1]{\textit{#1}}
\setlength{\parskip}{1em}
\setlength{\jot}{1em}
\addto\captionsenglish{\renewcommand{\contentsname}{Concepts}}

\hypersetup {
    linkcolor=black
}

% Create title
\title{
    \textbf{Complex Numbers:\\ Concepts, Formulas, etc.}
    \author{Alexandru Stan}
    \date{May 2024}
}

\begin{document}

    \maketitle
    \vfill
    \newpage

    \tableofcontents
    \fline
    \newpage

    \section{Preface}

    These notes are written as an assignment for AP Calculus AB and are meant to be used by myself as a tool 
    in any future university level mathematics class. If you are finding (or have received) these notes 
    and are wishing to learn/study from them; be aware that they are tailored to myself and may not 
    cover all concepts needed to properly learn/study complex numbers

    \section{Definition and Basics}

    To begin, a complex number is one where an ``imaginary'' part is present, 
    which will be defined as $i$ where $i=\sqrt{-1}$ moving onwards. 

    The imaginary part is in quotations as $i$ is not ``imaginary'' but instead
    on a different plane comapred to our traditional number line. They are 
    better refered to as lateral numbers but they will be referred to as 
    ``imaginary'' numbers to better reflect modern mathematical vocabulary

    \subsection{Parts of a Complex Number}

    % A complex number is defined as a number made up of a combination
    % of a real and imaginary part. 

    % \[
    %     z = a + bi    
    % \]

    % where $a$ and $b$ are both real numbers

    A complex number $z$ can be defined by that addition of a real part $a$ and imaginary part
    $bi$ such that $z = a + bi$ where $a, b\in\R$. 

    I'll cover this in the next subsection, but it is important to note that $0$ is also 
    a real number and that real numbers can also be written in complex form. 


    \subsection{Set Notation}


    Given $z$ where $z = a + bi$ such that $a, b \in\R$; $z$ can be said to be in the set of
    complex numbers $\C$ where $\C = \{a + bi \mid a, b\in\R, i^2 = -1\}$. This can also be written
    as $z\in\C$. 

    It is also important to note that all reals $\R$ are a subset of $\C$. This is inherently 
    true as any real number $x$ can be written as a complex number $a + bi$ where $b = 0$. 

    \section{Arithmetic Operations}

    All complex numbers abide by their respective arithemtic rules in such a way 
    where any arithmetic performed on a complex number returns another complex number 
    as shown below

    \begin{itemize}
        
        \item \bf Addition 
            $(a + bi) + (c + di) = (a + c) + i(b + d)$
        \item \bf Substraction 
            $(a + bi) - (c + di) = (a - c) + i(b - d)$
        \item \bf Multiplication
            $(a + bi)(c + di) = (ac - bd) + i(ad + bc)$
        \item \bf Division
            $\frac{a + bi}{(c + di)} = \frac{(ac + bd)}{c^2 + d^2} + \frac{i(bc - ad)}{c^2 + d^2}$

    \end{itemize}

    \section{Graphical Representation}

    Although often imagined only in an algebraic context, imaginary numbers can also be visually represented 
    in what can be understood as a modified cartesian plane. 

    \subsection{Complex Plane \& Point Representation}

    Given a complex number $z: z=x+yi$, the y-axis on the complex plane illustrates
    the imaginary component whilst the x-axis (also referred to as the number line) 
    represents the real component. 

    Below is a representation of the complex number $4 + 5i$ in point representation, i.e, 
    the real coefficients are treated as coordinates $(x, y)$ on the complex plane. 

    \begin{center}
        
        \begin{tikzpicture}
            % Draw axes
            \draw[->] (-1,0) -- (6,0) node[right] {Re};
            \draw[->] (0,-1) -- (0,5) node[above] {Im};
        
            % Draw grid lines
            \draw[dashed, gray] (5, 0) -- (5, 4);
            \draw[dashed, gray] (0, 4) -- (5, 4);
        
            % Label points
            \foreach \x in {1, 2, 3, 4, 5}
                \draw (\x, 0.1) -- (\x, -0.1) node[below] {\x};
        
            \foreach \y in {1, 2, 3, 4}
                \draw (0.1, \y) -- (-0.1, \y) node[left] {\y};
        
            % Draw and label the point (5, 4)
            \filldraw[black] (5, 4) circle (2pt) node[above right] {$5 + 4i$};

        \end{tikzpicture}

    \end{center}

    \subsection{Magnitude and Angle}

    Given a complex number $z$ such that $z = a + bi$ the magnitude of
    the complex number can be found by square rooting the sum of the squares
    of the number's real parts, i.e

    \[
        |z| = \sqrt{a^2 + b^2}    
    \]

    To find the angle of a complex number, arctangent can not be used due to the fact that if $\tan z = i$ and $\sin z = i\cos z$, 
    
    \[
        \begin{aligned}
            & \sin^2 z + \cos^2 z\\
            &= (i\cos z)^2 + \cos^2 z \\
            &= i^2\cos^2 z + \cos^2 z \\
            &= -\cos^2 z + \cos^2 z\\
            &= 0   
        \end{aligned}    
    \]

    but, $\sin^2 z + \cos^2 z = 1$ therefore there exists no solution to $\tan z = \pm i$. So instead
    the angle is findable using the real terms in the number such that 

    \[
        \tan\theta = \frac{b}{a}   
    \]

    \subsection{Vectors}

    Complex numbers are not identical to $\R^2$ but in some ways they 
    exhibit similar behaviours to each other. There are also similarities between complex
    number and two dimensional matrices. 

    There are two things you can do with a pair of complex numbers. You can add (or substract)
    and you can multiply (or divide) them. Thinking about addition and substraction, supposed 
    you map each complex number to a two dimensional vector as follow $a + bi \mapsto (a, b)$

    Then in term of complex numbers $(a + bi) + (c + di) = (a + c) + i(b + d)$ and in terms of vectors
    $(a, b) + (c, b) = (a + c, b + d)$ it can be observed that $(a + ib) + (c + ib) \mapsto (a, b) + (c, d)$   
    
    \newpage

    Now, thinking about multiplication and division, when we multiply a complex number by $i$ we
    get $i(c + id) = -d + ic$. 
    
    In the vector space you start with the vector $(c, d)$ and end up with a 
    vector $(-d, c)$. The vector has been rotated anticlockwise by $90^{\circ}$, representable via 
    the last matrice below
    
    \[
        \begin{pmatrix}
            0 & -1 \\
            1 & 0 \\
        \end{pmatrix}    
    \]

    We can now come up with a mapping from complex numbers to two dimensional matrices such that

    \[
        a + ib \mapsto \begin{pmatrix}
            a & -b \\
            b & a
        \end{pmatrix}
    \]

    That gives us $(a + ib)(c + id) = (ac - bd) + i(ad + bc)$ and

    \[
        \begin{pmatrix}
            a & -b\\
            b & a \\
        \end{pmatrix}
        \begin{pmatrix}
            c & -d \\
            d & c \\
        \end{pmatrix} = 
        \begin{pmatrix}
            ac-bd & -ad-bc\\
            ad + bc & ac -bd 
        \end{pmatrix}
    \]

    therefore

    \[
        (a + ib)(c + ib) \mapsto \begin{pmatrix}
            a & -b \\
            b & a
        \end{pmatrix}  
        \begin{pmatrix}
            c & -d \\
            d & c
        \end{pmatrix}
    \]

    We have found that the addition/substraction and multplication/division of 
    complex numbers and $\R^2$ vectors are isomorphic, i.e, similar in terms of 
    properties to one another. 

    \section{Complex Conjugate}
    \subsection{Definition}
    \subsection{Properties}

    \section{Magnitude (Modulus)}
    \subsection{Formula}
    \subsection{Properties}

    \section{Argument (Phase)}
    \subsection{Definition}
    \subsection{Principal Value}
    \subsection{Argument Function}
    \subsection{Properties}

    \section{Polar Form}
    \subsection{Conversion}
    \subsection{Formula}
    \subsection{Euler's Formula}
    \subsection{Inverse Conversion}

    \section{De Moivre's Theorem}
    \subsection{Definition}
    \subsection{Application}

    \section{Roots of Complex Numbers}
    \subsection{nth Roots}
    
    \section{Exponential Form and Logarithms}
    \subsection{Exponential Form}
    \subsection{Euler's Formula (with Logarithms)}
    \subsection{Logarithm}
    \subsection{Principal Value (with Logarithms)}

    \section{Functions of Complex Variables}
    \subsection{Exponential Function}
    \subsection{Trigonometric Functions}
    \subsection{Hyperbolic Functions}

    \section{Advanced Concepts}
    \subsection{Analytic Functions}
    \subsection{Cauchy-Riemann Equations}
    \subsection{Complex Integration}
    \subsection{Laplace Transform}



\end{document}