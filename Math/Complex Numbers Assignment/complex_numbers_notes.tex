\documentclass[12pt]{article}

% Import dependencies
\usepackage[utf8]{inputenc}
\usepackage[english]{babel}
\usepackage{indentfirst}
\usepackage{graphicx}
\usepackage{caption}
\usepackage{float}
\usepackage[margin=1.2in]{geometry} % margins
\usepackage{multicol}
\usepackage{wrapfig}
\usepackage{amsmath}
\usepackage{amssymb}
\usepackage[colorlinks=true, linkcolor=blue, urlcolor=cyan]{hyperref}
\usepackage{fancyhdr}
\usepackage{tikz}

\pagestyle{fancy}

\fancyhead[L]{\bf\large AP Calc - Complex Numbers}
\fancyhead[R]{\bf\large Sem. 2}
\setlength{\headheight}{35pt}

\newcommand{\C}{{\mathbb{C}}} 
\newcommand{\N}{{\mathbb{N}}}
\newcommand{\Q}{{\mathbb{Q}}}
\newcommand{\R}{{\mathbb{R}}}
\newcommand{\Z}{{\mathbb{Z}}}
\newcommand{\F}{{\mathbb{F}}}

\DeclareMathOperator{\arcsec}{arcsec}
\DeclareMathOperator{\arccot}{arccot}
\DeclareMathOperator{\arccsc}{arccsc}
\DeclareMathOperator{\arctantwo}{arctan2}

\renewcommand{\baselinestretch}{1.5}
\newcommand{\fline}{\par\noindent\rule{\textwidth}{0.1pt}}
\newcommand{\uit}[1]{\textit{#1}}
\setlength{\parskip}{1em}
\setlength{\jot}{1em}
\addto\captionsenglish{\renewcommand{\contentsname}{Concepts}}

\hypersetup {
    linkcolor=black
}

% Create title
\title{
    \textbf{Complex Numbers:\\ Concepts, Formulas, etc.}
    \author{Alexandru Stan}
    \date{May 2024}
}

\begin{document}

    \maketitle
    \vfill
    \newpage

    \tableofcontents
    \fline
    \newpage

    \section{Preface}

    These notes are written as an assignment for AP Calculus AB and are meant to be used by myself as a tool 
    in any future university level mathematics class. If you are finding (or have received) these notes 
    and are wishing to learn/study from them; be aware that they are tailored to myself and may not 
    cover all concepts needed to properly learn/study complex numbers

    \section{Definition and Basics}

    To begin, a complex number is one where an ``imaginary'' part is present, 
    which will be defined as $i$ where $i=\sqrt{-1}$ moving onwards. 

    The imaginary part is in quotations as $i$ is not ``imaginary'' but instead
    on a different plane comapred to our traditional number line. They are 
    better refered to as lateral numbers but they will be referred to as 
    ``imaginary'' numbers to better reflect modern mathematical vocabulary

    \subsection{Parts of a Complex Number}

    % A complex number is defined as a number made up of a combination
    % of a real and imaginary part. 

    % \[
    %     z = a + bi    
    % \]

    % where $a$ and $b$ are both real numbers

    A complex number $z$ can be defined by that addition of a real part $a$ and imaginary part
    $bi$ such that $z = a + bi$ where $a, b\in\R$. 

    I'll cover this in the next subsection, but it is important to note that $0$ is also 
    a real number and that real numbers can also be written in complex form. 


    \subsection{Set Notation}


    Given $z$ where $z = a + bi$ such that $a, b \in\R$; $z$ can be said to be in the set of
    complex numbers $\C$ where $\C = \{a + bi \mid a, b\in\R, i^2 = -1\}$. This can also be written
    as $z\in\C$. 

    It is also important to note that all reals $\R$ are a subset of $\C$. This is inherently 
    true as any real number $x$ can be written as a complex number $a + bi$ where $b = 0$. 

    \section{Arithmetic Operations}

    All complex numbers abide by their respective arithemtic rules in such a way 
    where any arithmetic performed on a complex number returns another complex number 
    as shown below

    \begin{itemize}
        
        \item \bf Addition 
            $(a + bi) + (c + di) = (a + c) + i(b + d)$
        \item \bf Substraction 
            $(a + bi) - (c + di) = (a - c) + i(b - d)$
        \item \bf Multiplication
            $(a + bi)(c + di) = (ac - bd) + i(ad + bc)$
        \item \bf Division
            $\frac{a + bi}{(c + di)} = \frac{(ac + bd)}{c^2 + d^2} + \frac{i(bc - ad)}{c^2 + d^2}$

    \end{itemize}

    \section{Graphical Representation}

    Although often imagined only in an algebraic context, imaginary numbers can also be visually represented 
    in what can be understood as a modified cartesian plane. 

    \subsection{Complex Plane \& Point Representation}

    Given a complex number $z: z=x+yi$, the y-axis on the complex plane illustrates
    the imaginary component whilst the x-axis (also referred to as the number line) 
    represents the real component. 

    Below is a representation of the complex number $4 + 5i$ in point representation, i.e, 
    the real coefficients are treated as coordinates $(x, y)$ on the complex plane. 

    \begin{center}
        
        \begin{tikzpicture}
            % Draw axes
            \draw[->] (-1,0) -- (6,0) node[right] {Re};
            \draw[->] (0,-1) -- (0,5) node[above] {Im};
        
            % Draw grid lines
            \draw[dashed, gray] (5, 0) -- (5, 4);
            \draw[dashed, gray] (0, 4) -- (5, 4);
        
            % Label points
            \foreach \x in {1, 2, 3, 4, 5}
                \draw (\x, 0.1) -- (\x, -0.1) node[below] {\x};
        
            \foreach \y in {1, 2, 3, 4}
                \draw (0.1, \y) -- (-0.1, \y) node[left] {\y};
        
            % Draw and label the point (5, 4)
            \filldraw[black] (5, 4) circle (2pt) node[above right] {$5 + 4i$};

        \end{tikzpicture}

    \end{center}

    \subsection{Vectors}

    Complex numbers are not identical to $\R^2$ vectors but in some ways they 
    exhibit similar behaviours to each other. There are also similarities between complex
    numbers and two dimensional matrices. 

    There are two things you can do with a pair of complex numbers. You can add (or substract)
    and you can multiply (or divide) them. Thinking about addition and substraction, suppose 
    you map each complex number to a two dimensional vector as follows $a + bi \mapsto (a, b)$

    Then in term of complex numbers $(a + bi) + (c + di) = (a + c) + i(b + d)$ and in terms of vectors
    $(a, b) + (c, b) = (a + c, b + d)$ it can be observed that $(a + ib) + (c + ib) \mapsto (a, b) + (c, d)$   

    Now, thinking about multiplication and division, when we multiply a complex number by $i$ we
    get $i(c + id) = -d + ic$. 
    
    In the vector space you start with the vector $(c, d)$ and end up with a 
    vector $(-d, c)$. The vector has been rotated anticlockwise by $90^{\circ}$, representable via 
    the last matrice below
    
    \[
        \begin{pmatrix}
            0 & -1 \\
            1 & 0 \\
        \end{pmatrix}    
    \]

    We can now come up with a mapping from complex numbers to two dimensional matrices such that

    \[
        a + ib \mapsto \begin{pmatrix}
            a & -b \\
            b & a
        \end{pmatrix}
    \]

    That gives us $(a + ib)(c + id) = (ac - bd) + i(ad + bc)$ and

    \[
        \begin{pmatrix}
            a & -b\\
            b & a \\
        \end{pmatrix}
        \begin{pmatrix}
            c & -d \\
            d & c \\
        \end{pmatrix} = 
        \begin{pmatrix}
            ac-bd & -ad-bc\\
            ad + bc & ac -bd 
        \end{pmatrix}
    \]

    therefore

    \[
        (a + ib)(c + ib) \mapsto \begin{pmatrix}
            a & -b \\
            b & a
        \end{pmatrix}  
        \begin{pmatrix}
            c & -d \\
            d & c
        \end{pmatrix}
    \]

    We have found that the addition/substraction and multplication/division of 
    complex numbers and $\R^2$ vectors are isomorphic, i.e, similar in terms of 
    properties to one another. 

    \section{Complex Conjugate}
    \subsection{Definition}

    A complex conjugate is one where, given a complex number $z: z = a + bi$ the sign 
    of the imaginary component is flipped. This action returns a complex number with a 
    structure of $a - bi$. The complex conjugate is often noted as $\bar{z}$

    \subsection{Properties}
        
    The following properties apply for all complex numbers $z$ and $w$ unless stated otherwise, and \
    can be proved by writing $z$ and $w$ in the form $a + bi$

    For any two complex numbers, conjugation is distributive over addition, substraction, multiplication, and division

    \[
        \begin{aligned}
            \overline{z + w} &= \overline{z} + \overline{w}\\
            \overline{z - w} &= \overline{z} - \overline{w}\\
            \overline{zw} &= \overline{z}\overline{w}\\
            \overline{\left(\frac{z}{w}\right)} &= \frac{\overline{z}}{\overline{w}}, \space\text{if} \space\ne0
        \end{aligned}    
    \]

    A complex number is equal to its complex conjugate if its imaginary part, $b$, is equal to $0$. In other words, real numbers
    are the only fixed points of conjugation
    Conjugation does not change the modulus of a complex nmumber: $|\overline{z}| = |z|$. Conjugation is also an involution, that is, 
    the conjugate of the conjugate of a complex number $z$ is $z$, also written as $\overline{\overline{z}} = z$

    \section{Modulus}

    Given a complex number $z$ such that $z = a + bi$ the modulues (also known as the magnitude) of
    the complex number can be found by square rooting the sum of the squares
    of the number's real parts, i.e

    \[
        |z| = \sqrt{a^2 + b^2}    
    \]

    The modulus is often written as $r$ where $r = |z|$

    \section{Argument}

    \subsection{Definition}

    For a complex number $z = a + bi$ where $a, b \in\R$ the argument $\theta$ is the angle formed between
    the positive real axis (defined as $1 + 0i$) and the line representing the complex number $z$. 
    
    \subsection{Mathematical Expression}

    The argument can be found using the arctangent function such that $\theta = \arctan\left(\frac{b}{a}\right)$

    However, since arctangent alone only returns results in the range $(-\frac{\pi}{2}, \frac{\pi}{2})$, adjustements 
    are needed to place $\theta$ correctly in all four quadrants. This leads to the use of the $\arctantwo(b, a)$ function,
    which returns the angle in the correct quadrant

    \subsection{Range of Values}

    The argument $\theta$ can theoretically take on a real value, but it is often restricted to a principal value 
    to ensure that it is unique. The principal value of the argument is typically within the interval $-\pi < \theta\le\pi$
    
    Note that, $Arg(z)$ restricts the angle in the above interval while $\arg(z)$ does not and can be defined as 
    $\arg(z) = \{Arg(z) + 2\pi n \mid n \in\N\}$ where $Arg(z) = \arctantwo(z)$

    \subsection{Principal Argument}

    The principle argument, denoted $Arg(z)$, is the unique value of the argument within the specified interval 
    $Arg(z) = arg(z) \mod 2\pi$ with $-\pi < Arg(z)\le\pi$

    \section{Polar Form}

    The polar form of a complex number is a different way to represent a plex number apart from the 
    traditional rectangular form. Usually a complex numer is represented in the form $a + bi$, but in 
    polar form, it is representated as a combination of the modulus (magnitude) and argument.

    \subsection{Conversion}

    Converting a complex number from rectangular form to polar form involved finding the modulus and the argument. 
    Polar form is represented as $z = r(\cos\theta + i\sin\theta)$ or more commonly as $z = re^{i\theta}$. Below are 
    steps to follow to perform the Conversion
    
    \begin{enumerate}
        
        \item Calculate the modulus
        
        \[
            r = \sqrt{a^2 + b^2}    
        \]

        \item Calculate the argument
        
        

    \end{enumerate}

    \subsection{Formula}
    \subsection{Euler's Formula}
    \subsection{Inverse Conversion}
    \subsection{Properties}

    \section{De Moivre's Theorem}
    \subsection{Definition}
    \subsection{Application}

    \section{Roots of Complex Numbers}
    \subsection{nth Roots}
    
    \section{Exponential Form and Logarithms}
    \subsection{Exponential Form}
    \subsection{Euler's Formula (with Logarithms)}
    \subsection{Logarithm}
    \subsection{Principal Value (with Logarithms)}

    \section{Functions of Complex Variables}
    \subsection{Exponential Function}
    \subsection{Trigonometric Functions}
    \subsection{Hyperbolic Functions}

    \section{Advanced Concepts}
    \subsection{Analytic Functions}
    \subsection{Cauchy-Riemann Equations}
    \subsection{Complex Integration}
    \subsection{Laplace Transform}



\end{document}