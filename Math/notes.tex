\documentclass[12pt]{article}

% Import dependencies
\usepackage[utf8]{inputenc}
\usepackage[english]{babel}
\usepackage{indentfirst}
\usepackage{graphicx}
\usepackage{caption}
\usepackage{float}
\usepackage[margin=1.2in]{geometry} % margins
\usepackage{multicol}
\usepackage{wrapfig}
\usepackage{amsmath}
\usepackage{amssymb}
\usepackage[colorlinks=true, linkcolor=blue, urlcolor=cyan]{hyperref}
\usepackage{fancyhdr}

\pagestyle{fancy}

\fancyhead[L]{\bf\large AP Calc - Notes}
\fancyhead[R]{\bf\large Semester 2}
\setlength{\headheight}{35pt}

\newcommand{\C}{{\mathbb{C}}} 
\newcommand{\N}{{\mathbb{N}}}
\newcommand{\Q}{{\mathbb{Q}}}
\newcommand{\R}{{\mathbb{R}}}
\newcommand{\Z}{{\mathbb{Z}}}
\newcommand{\F}{{\mathbb{F}}}

\renewcommand{\baselinestretch}{1.5}
\newcommand{\fline}{\par\noindent\rule{\textwidth}{0.1pt}}
\newcommand{\uit}[1]{\textit{#1}}
\setlength{\parskip}{1em}
\addto\captionsenglish{\renewcommand{\contentsname}{Units}}

\hypersetup {
    linkcolor=black
}

% Create title
\title{
    \textbf{AP Calculus AB:\\ Notes, Formulas, Examples}
    \author{Alexandru Stan}
    \date{Start date: February 2, 2024 \\ End date: May 13, 2024}
}

\begin{document}

    \maketitle
    \vfill
    \begin{center}
        \uit{
            Sections based off of Colleged Board units and Mrs. Cooper's Lessons. \\
            Formatting may vary and be of differ in quality
        }
    \end{center}
    \newpage

    \tableofcontents
    \fline
    \newpage

    \section{Limits and Continuity}
    \fline

    The limit is when a given value approaches, or gets \textit{really close} (infinitely) to another value. 
    The standard limit notation is:

    \[
        \lim_{x \to c} f(x)    
    \]

    represents when $x$ can approach $c$ from either left ($-$) or the right ($+$). By adding
    a sign superscript to the $c$, it means that $x$ can only approach from that direction:

    \[
        \lim_{x \to c^+} f(x)    
    \]

    \begin{center}
        \uit{Right hand limit}, $x$ approaches $c$ from values greater than $c$
        \[
            \lim_{x \to c^-} f(x)    
        \]
        \uit{Left hand limit}, $x$ approaches $c$ from values lower than $c$
    \end{center}

    \subsection{Limits to Infinity}

    If a degree (biggest exponent) of a polynomial is greater than or equal to $1$, its
    limit as $x$ approaches $\pm\infty$ will also be $\pm\infty$. This depends on the sign of the leading
    coefficient and the degree of polynomial

    \noindent Example:

    \[
        \begin{aligned}
            f(x) &= 3x^3 - 7x^2 + 2 \\
            \lim_{x \to \infty} f(x) &= \infty \\
            \lim_{x \to -\infty} f(x) &= -\infty     
        \end{aligned}    
    \]

    The degree of $f(x)$ is 3, and the leading coefficient is positive. The graph
    goes down to up from left to right.

    With Fractions, just find whether the highest deree is the numerator or the denominator. 
    Numerator means $\infty$, denominator means $0$

    \subsection{Asympotes}
    Functions can have asymptotes, either vertical or horizontal. In the case of vertical asymptotes, the limit
    would be \uit{unbounded} as it approaches that $x$ value. 

    \noindent Example:

    \[
        \begin{aligned}
            f(x) &= \frac{2x-4}{x-3}\\
            \lim_{x \to 3} f(x) &= \text{undef} \\
            \lim_{x \to 3^-} f(x) &= -\infty \\
            \lim_{x \to 3^+} f(x) &= \infty 
        \end{aligned}
    \]

    As with vertical asymptotes, as $x$ approaches $c$ (in this case $\pm\infty$), the limit would
    approach the horizontal asymptote. Although the $y$-value never actually touches the 
    asymptote, the limit gets really close to the value, from both below and above

    \subsection{Limit Properties}

    The limits of combined functions can be found by finding the limit of each of the 
    individual functions, then applying the operations.

    \begin{itemize}
        
        \item \textbf{Addition/Substraction}
        
        When taking the limit of the sum or difference of multiple functions, it's the 
        same thing as taking the sum of difference of each of the seperate limits of each function
        
        \[
            \lim_{x \to c} [f(x) + g(x)] \implies \lim_{x \to c} f(x) + \lim_{x \to c} g(x)   
        \]

        \[
            \lim_{x \to c} [f(x) - g(x)] \implies \lim_{x \to c} f(x) - \lim_{x \to c} g(x)
        \]

        Note that when the limit of either function is \uit{undefined} the combined
        limit would also be undefined

        \item \textbf{Multiplication}
        
        Multiplication of the limits of functions is quite straightforward
        
        \[
            \lim_{x \to c} [f(x) \cdot g(x)] \implies \lim_{x \to c} f(x) \cdot \lim_{x \to c} g(x)    
        \]

        The same exception applies when one of the limits is \uit{undefined}. This just makes
        the entire combined limit undefined

        \item \textbf{Division}
        
        Division is basically the same as the other basic operations except if the denominator is 0

        \[
            \lim_{x \to c} \frac{f(x)}{g(x)} \implies \frac{\lim_{x \to c} f(x)}{\lim_{x \to c} g(x)}   
        \]

        \item \textbf{Composite Functions}
        
        When working with composite functions, it's the same thing as taking
        the limit of the inner function, then evaluating the outer function normally

        \[
            \lim_{x \to c} f\bigg(g(x)\bigg) \implies f\bigg(\lim_{x \to c} g(x)\bigg)    
        \]

        \item \textbf{Other Theorems}
        
        Given that $\lim f(x)$ and $\lim g(x)$ are both finite for all numbers, and $C$ is a ``constant"

        \[
            \begin{aligned}
                \lim kf(x) &= k \lim f(x) \\
                \lim_{x \to a} C &= C 
            \end{aligned}    
        \]

    \end{itemize}

    \subsection{Solving Limits}

        The first thing to always try to do when solving limits is \textbf{direct substitution}. If this
        is not possible (undefined limit), then algebraic manipulation (factoring) is the next step

        \[
            \begin{aligned}
                & lim_{x \to c} \frac{x^4 + 3x^3 - 10x^2}{x^2 - 2x} \\
                &= \lim_{x \to can} \frac{x^2(x^2 + 3x - 10)}{x(x - 2)} \\
                &= \lim_{x \to c} \frac{x^2(x+5)(x-2)}{x(x-2)}\\
                &= \lim_{x \to c} x^2(x+5)\\
            \end{aligned}    
        \]

        When encountering radicals, conjugates can be used.

        \[
            \begin{aligned}
                & \lim_{x \to c} \frac{x + 4}{\sqrt{3x + 13} - 1} \\
                &= \lim_{x \to c} \frac{x + 4}{\sqrt{3x + 13} - 1} \cdot \frac{\sqrt{3x + 13} + 1}{\sqrt{3x + 13} + 1} \\
                &= \lim_{x \to c} \frac{(x + 4)(\sqrt{3x + 13} + 1)}{3x + 12} \\
                &= \lim_{x \to c} \frac{(x + 4)(\sqrt{3x + 13} + 1)}{3(x + 4)} \\
                &= \lim_{x \to c} \frac{\sqrt{3x + 13} + 1}{3}
            \end{aligned}    
        \]

        When dealing with trigonometric equations, trig identities can be used
        (assuming direct substitution doesn't work)

        \[
            \begin{aligned}
                &\lim_{x \to c} \frac{\cot^2(x)}{1 - \sin(x)} \\
                &= \lim_{x \to c} \frac{\cos^2(x)}{(\sin^2(x))(1 - \sin(x))} \\
                &= \lim_{x \to c} \frac{1 - \sin^2(x)}{(\sin^2(x))(1 - \sin(x))} \\
                &= \lim_{x \to c} \frac{(1 + \sin(x))(1 - \sin(x))}{(\sin^2(x))(1 - \sin(x))} \\
                &= \lim_{x \to c} \frac{1 + \sin(x)}{\sin^2(x)}, \text{for x} \ne (2k + 1)\frac{\pi}{2}
            \end{aligned}    
        \]

        However, functions can not always be factored, so in that case they will just be undefined

        \[
            \begin{aligned}
                & \lim_{x \to 1} \frac{2x}{x^2 - 7x + 6} \\
                &= \lim_{x \to 1} \frac{2x}{(x -6)(x -1)} \\
                &= \frac{2}{0} \\
                &= \text{undef}
            \end{aligned}    
        \]

        \subsection{Continuity}

        A function is continous at a point if its right and left hand side limit at that point are the same. 
        In other words, it can be drown without lifting the pencil.

        \[
            \lim_{x \to c^-} f(x) = \lim_{x \to c^+} f(x) = f(c)    
        \]

        For a function $f$ to be continous for all $\R$ it has to return a real numbe result for all 
        real number values of $x$. Basically $f: \R \to \R$ 

        \begin{itemize}
            
            \item $\sqrt{x + 4}$ is continous $\forall x: x \ge -4$
            \item $\sqrt[5]{x}$ is continous $\forall x: x \in \R$
            \item $\ln x$ is continous $\forall x: x > 0$
            \item $\frac{1}{x-3}$ is continous $\forall x: x \ne 3$

        \end{itemize}

        \textbf{Removable dicontinuity} is function, where a point is ``removed'', and the graph of the new function 
        is almost identical to the original function
        
        \noindent Given:

        \[
            \lim_{x \to c} f(x) = k \le \infty
        \]

        \noindent where:

        \[
            F(x) = \begin{cases}
                f(x) & \text{if } x \ne c \\
                k & \text{if } x = c
            \end{cases}
        \]

        then $F(x)$ has a removable dicontinuity at $k$

        \textbf{Jump dicontinuity} is when the graph jumps from one $y$ value to another at the 
        same $x$-value. 

        \textbf{Infinite dicontinuity} Usually occurs when there is a vertical asymptote, and the 
        dicontinuity occurs over asymptote. Basically, both sides of the asymptote approach that $x$-value, but never actually touch, 
        so the function is not continous. 

        \subsection{Squeeze Theorem}

        When it is to find the limit for a function, the squeeze theorem can be used. Basically, you find
        two other functions, one on top and on below, and use their limits to ``squeeze'' the limit of 
        the given function. 

        \noindent Given:

        \[
            g(x) \le f(x) \le h(x) 
        \]

        \noindent for all $x$ in an open interval that includes $c$, and 

        \[
            \lim_{x \to c} f(x) = \lim_{x \to c} h(x) = L    
        \]
        
        \noindent then,
        \[
            \lim_{x \to c} g(x) = L
        \]
        
        \noindent Note that $x$ and $L$ can both be $\pm\infty$

        \noindent \textbf{Example:}

        \[
            \begin{aligned}
                & \text{Problem:}                          &                                  & \lim_{x \to \infty} \frac{\sin{x}}{x} \\[6pt]
                & \text{keep in mind that}                 & -1                               & \le \sin{x} \le 1                     \\
                & \text{divide by $x$}                     & \frac{-1}{x}                     & \le \frac{\sin{x}}{x} \le \frac{1}{x} \\[6pt]
                & \text{take limits of smaller functions } & \lim_{x \to \infty} \frac{-1}{x} & = 0 = \lim_{x \to \infty} \frac{1}{x} \\[6pt]
                & \text{Squeeze Theorem:}                  & \lim_{x \to \infty}              & \frac{\sin{x}}{x} = 0.
                \end{aligned}    
        \]

        The best way to solve the above problem is to recognmize the easier part of the problem, 
        in this case $\sin x$, then manipulate the inequality in such a way that the middle function becomes the 
        original problem. Solve the limits of the other two functions to solve the original limit

        \subsection{Intermediate Value Theorem (IVT)}

        Given a function $f$ where $f \in C[a, b]$ and $c \in [a, b]$. Then there must a value  $c$ such that $f(a) \le f(c) \le f(b)$. In other words, if a 
        function is continous from $a \to b$, then it must take on every value between $f(a)$ and $f(b)$ for all values of $x$ 
        such that $x \in [a, b]$

        \section{Differentiation: Definition and Fundamental}

        A derivative is the \textbf{instantaneous} rate of change of a function at a point. It's the average
        rate of change over an infintely small interval. It has two main notations. 

        \begin{itemize}
            
            \item \textbf{Lagrange's Notation: } The derivative of $f(x)$ is denoted as $f'(x)$, pronounced as ``f prime of x''. 
                                                 Higher order derivatives are denoted as $f''(x)$ or $f^2(x)$, etc. In general it's 
                                                 written as $f^{n}(x)$, or with the $n$ in ticks.

            \item \textbf{Leibniz's Notation: } The derivative of $f(x)$ is denoted as $\frac{dy}{dx}$, pronounced as ``dee y over dee x''. 
                                                Higher order derivatives are denoted as $\frac{d^2y}{dx^2}$, etc. In general it's 
                                                written as $\frac{d^ny}{dx^n}$

        \end{itemize}

        \subsection{Continuity and Differentiability}

        \noindent\textbf{Differentiability:} A function is differentiable for every value in its domain \\
        \noindent\textbf{Continuity:} The function has no breaks over its domain, can be drawn without lifting the pencil

        \noindent Differentiability \textit{implies} continuity, but not the other way around

        \subsection{Derivative as a Limit}

        The derivative of a function $f(x)$ at a point $x = a$ is quite truly just first principles, it is as follows:

        \[
            \begin{aligned}
                \frac{d}{dx} f(x) &= \lim_{h \to 0} \frac{f(x + h) - f(x)}{h} \\
                \frac{d}{dx} f(a) &= \lim_{x \to a} \frac{f(x) - f(a)}{h}
            \end{aligned}
        \]

        \subsection{Differentiation Rules}

        \subsubsection{Derivative of a Constant}

        The derivative of a constant is always $0$. This is because the slope of a constant function is always $0$

        \[
            \begin{aligned}
                f(x) &= C \\
                f^{\prime}(x) &= C^{\prime} = 0
            \end{aligned}    
        \]

        \subsubsection{Constant in a function}

        The constant can be moved out in front of the derivative

        \[
            (k f(x))^{\prime} = k f^{\prime}(x)
        \]

        \subsubsection{Sum Rule}

        The derivative of the sum of many functions is the same as the sum of the derivatives
        of the individual functions. The same applies for subtraction

        \[
            \sum_{k = 1}^n \bigg[f_n(x)\bigg] = \left[\sum_{k = 1}^n \bigg[f_n(x)\bigg]\right]^{\prime}
        \]

        \subsubsection{Power Rule}

        You put the exponent in front of the function as constant and then substract 1 from the exponent.
        This also applies to negative or fractional exponents (radicals)

        \[
            \begin{aligned}
                f(x) &= x^n : n \in \R \\
                f^{\prime}(x) &= nx^{n-1}
            \end{aligned}    
        \]

        \subsubsection{Product Rule}

        \[
            \bigg[f(x) \cdot g(x)\bigg]^{\prime} = f^{\prime}(x)g(x) + f(x)g^{\prime}(x)    
        \]

        \subsubsection{Quotient Rule} 

        \[
            \bigg[\frac{f(x)}{g(x)}\bigg]^{\prime} = \frac{f^{\prime}(x)g(x) - f(x)g^{\prime}(x)}{g^2(x)}    
        \]

        \subsubsection{Chain Rule}

        The chain rule allows for the differentiation of a \textit{composition} of two or more function. 
        Take the derivative of the inner function, then multiply that by the derivative of the outer function. 

        \[
                \frac{d}{dx} \bigg[f(g(x))\bigg] = f^{\prime}(g(x)) \cdot g^{\prime}(x)
        \]

        \subsection{Exponential Functions}

        Can be solved like the \textbf{chain rule}, with the base and exponent as the outer and inner
        functions, respectively. The formula is:

        \[
            \frac{d}{dx}(a^x) = a^x \cdot \ln a    
        \]

        \noindent The only exception is $e^x$

        \[
            \frac{d}{dx} (e^x) = e^x   
        \]

        \subsection{Lograithmic Functions}

        The derivative of $\ln x$ is:
        
        \[
            \frac{d}{dx} (\ln x) = \frac{1}{x}  
        \]

        This can be used to derive the derivative of other base log functions

        \[
            \begin{aligned}
                \frac{d}{dx} (\log_a x) &= \frac{d}{dx} \bigg(\frac{\ln x}{\ln a}\bigg) \\
                                        &= \frac{a}{\ln a} \cdot \frac{d}{dx}(\ln x) \\
                                        &= \frac{1}{\ln a} \cdot \frac{1}{x} \\
                                        &= \frac{1}{x\ln a}  
            \end{aligned}
        \]

        \subsection{Trigonometric Functions}

         There isn't really an eaasy way to memorize these, just do enough problems
         and you'll get the hang of it.

         Although, do note that all of the functions other than $\sin$ and $\cos$ can be derived using the 
         quotient of chain Rules

         
        \begin{center}
            \begin{tabular}{ | c |  c  |}
                \hline
                $f(x)$ & $f^{\prime}(x)$ \\
                \hline
                $\sin x$ & $\cos x$ \\
                \hline
                $\cos x$ & $-\sin x$ \\
                \hline
                $\tan x$ & $\sec^2 x$ \\
                \hline
                $\cot x$ & $-\csc^2 x$ \\
                \hline
                $\sec x$ & $\sec x \cdot \tan x$ \\
                \hline
                $\csc x$ & $-\csc x \cdot \cot x$ \\
                \hline
             \end{tabular}
        \end{center}

        \subsection{Inverse Functions}

        The derivative of an inverse function is the reciprocal of the 
        derivative of the original with it's input being the inverse

        \[
                \frac{d}{dx} f^{-1}(x) = \frac{1}{f^{\prime}(f^{-1}(x))}   
        \]

        \section{Differentiation: Composite, Implicit, and Inverse Functions}

        \subsection{Implicit Differentiation}

        Implicit differentiation is taking the derivative of both sides of an equation with respect to
        two variables, usually $x$ and $y$, by treating one of the variables as a function of the other. 
        (Usually $y$ is a function of $x$)

        \noindent Example:

        \[
            \begin{aligned}
                x^2 + y^2 &= 1 \\
                \frac{d}{dx} (x^2 + y^2) &= \frac{d}{dx} \\
                \frac{d}{dx} (x^2) + \frac{d}{dx} (y^2) &= 0 \\
                2x + 2y \cdot \frac{dy}{dx} &= 0 \\
                x + y \cdot \frac{dy}{dx} &= 0 \\
                \frac{dy}{dx} &= -\frac{x}{y}
            \end{aligned}
        \]

        When taking the derivative of $y^2$, multiply by $\frac{dy}{dx}$ because the equation is being taken
        as a function of $x$

        \subsection{Inverse Trigonometric Functions}

        These equations cna be found using implicit differentiation along with trig identities. 
        \textit{A more indepth explanation is provided for each under the table}

        \begin{center}
            \begin{tabular}{ | c | c |}
                \hline
                $f(x)$ & $f^{\prime}(x)$ \\
                \hline
                $\arcsin x$ & $\frac{1}{\sqrt{1 - x^2}}$ \\
                \hline
                $\arccos x$ & $-\frac{1}{\sqrt{1 - x^2}}$ \\
                \hline
                $\arctan x$ & $\frac{1}{1 + x^2}$ \\
                \hline
                $\text{arcsec } x$ & $\frac{1}{|x|\sqrt{x^2 - 1}}$ \\
                \hline
                $\text{arccsc } x$ & $-\frac{1}{|x|\sqrt{x^2 - 1}}$ \\
                \hline
                $\text{arccot } x$ & $-\frac{1}{1 + x^2}$ \\
                \hline
            \end{tabular}
        \end{center}

        \begin{itemize}
            
            \item arcsin $x$ 
            
                \[
                    \begin{aligned}
                        y = \arcsin x &\implies x = \sin y \\  
                        \frac{d}{dx}(\sin y) &= \frac{d}{dx} x \\
                        \frac{dy}{dx} (\cos y) &= 1 \\
                        \frac{dy}{dx} &= \frac{1}{\cos y} \\
                        \frac{dy}{dx} &= \frac{1}{\sqrt{1 - \sin^2 y}} \\
                        \frac{dy}{dx} &= \frac{1}{\sqrt{1 - x^2}} \\
                    \end{aligned}  
                \]

            \item arccos $x$

                \[
                    \begin{aligned}
                        y = \arccos x &\implies x = \cos y \\  
                        \frac{d}{dx}(\cos y) &= \frac{d}{dx} x \\
                        \frac{dy}{dx} (-\sin y) &= 1 \\
                        \frac{dy}{dx} &= \frac{-1}{\sin y} \\
                        \frac{dy}{dx} &= \frac{-1}{\sqrt{1 - \cos^2 y}} \\
                    \end{aligned}  
                \]

            \item arctan $x$ 
            
                \[
                    \begin{aligned}
                        y = \arctan x &\implies x = \tan y \\  
                        \frac{d}{dx}(\tan y) &= \frac{d}{dx} x \\
                        \frac{dy}{dx} (\sec^2 y) &= 1 \\
                        \frac{dy}{dx} &= \frac{1}{\sec^2 y} \\
                        \frac{dy}{dx} &= \frac{1}{1 + \tan^2 y} \\
                        \frac{dy}{dx} &= \frac{1}{1 + x^2} \\
                    \end{aligned}  
                \]

        \end{itemize}

        \subsection{Higher Order Derivatives}

        To find higher order derivatives, take the derivative of the prvious order derivative. 

        \[
            \frac{d^n}{dx^n} = \frac{d}{dx}\bigg(\frac{d^{n-1}}{dx^{n-1}} f(x)\bigg)   
        \]

        \section{Contextual Applications of Differentiation}

        \subsection{Straight Line Motion: Position, Velocity, and Acceleration}

        \noindent\textbf{Position: } where something is at a given time $t$

        \noindent\textbf{Velocity: } how fast something is moving at a given time $t$. Determines 
        the direction of the object

        \[
            v(t) \begin{cases}
                < 0 \implies \text{Left} \\
                = 0 \implies \text{Stopped} \\
                > 0 \implies \text{Right}
            \end{cases}    
        \]

        \noindent\textbf{Acceleration: } Determines whether the velocity is increasing at a given time $t$. 
        If it's sign is the same as that of the velocity, the object is speeding up. If the two signs are different, the
        object is slowing down. If the acceleration is 0, the object is moving at a constant speed. 

        Position, velocity, and acceleration are all related as follows

        \[
            \begin{aligned}
                x^{\prime}(t) &= v(t) \\
                v^{\prime}(t) &= a(t) \\
            \end{aligned}    
        \]

        \subsection{Related Rates}

        Related rates is using implicit differentiation and given variables to solve for unknown variables

        \noindent Examples:

        \begin{enumerate}
            
            \item Given the equation $\frac{x}{y} = 9$ and, $\frac{dy}{dt} = -\frac{2}{3}$, 
            find $\frac{dx}{dt}$ when $x = 3$

            \textbf{Solution: }

            First, differentiate $\frac{x}{y} = 9$ with respect to $t$

            \[
                \begin{aligned}
                    \frac{x}{y} &= 9 \\
                    \frac{y \cdot \frac{dx}{dt} - x \cdot \frac{dy}{dt}}{y^2} &= 0 \\
                \end{aligned}
            \]

            To solve for $\frac{dx}{dt}$, we first have to solve for $y$

            \[
                \begin{aligned}
                    \frac{x}{y} &= 9 \\
                    \frac{3}{y} &= 9 \\
                    3 &= 9y \\
                    y &= \frac{1}{3}
                \end{aligned}    
            \]

            Finally, plug in all the variables to solve for $\frac{dx}{dt}$

            \[
                \begin{aligned}
                    \frac{y \cdot \frac{dx}{dt} - x \cdot \frac{dy}{dt}}{y^2} &= 0 \\
                    \frac{\frac{1}{3} \cdot \frac{dx}{dt} - 3 \cdot -\frac{2}{3}}{\frac{1}{3^2}} &= 0 \\
                    \frac{\frac{1}{3} \cdot \frac{dx}{dt} + 2}{\frac{1}{3}} &= 0 \\
                    \frac{1}{3} \cdot \frac{dx}{dt} + 2 &= 0 \\
                    \frac{dx}{dt} &= -6
                \end{aligned}    
            \]

            \item The surface area of a sphere if increasing at a rate of $14\pi$ square meters per hour. 
            At a certain instant, the surface area is $36\pi$ square meters. \textbf{What is the rate of the 
            volume of the sphere at that instant (in cubic meters per hours)?}

            \textbf{Solution: }

            The surface area (A) of a sphere with radius $r$ is $4\pi r^2$ 
            The volume (V) of a sphere with radius $r$ is $\frac{4}{3}\pi r^3$
            First, identiy what was given:

            \begin{itemize}
                \item $\frac{dA}{dt} = 14\pi$
                \item $A = 36\pi = 4\pi r^2$
                \item $V = \frac{4}{3}\pi r^3$
            \end{itemize}

            Next, what is unknown:

            \begin{itemize}
                \item $r = ?$
                \item $\frac{dV}{dt} = ?$
            \end{itemize}

            Secondly, solve for for $r$

            \[
                \begin{aligned}
                    A &= 4\pi r^2 \\
                    36\pi &= 4\pi r^2 \\
                    \frac{36\pi}{4\pi} &= r^2 \\
                    \sqrt{9} &= r \\
                    r &= 3
                \end{aligned}
            \]

            Then, solve for $\frac{dr}{dt}$

            \[
                \begin{aligned}
                    A &= 4\pi r^2 \\
                    \frac{dA}{dt} &= 8\pi r \cdot \frac{dr}{dt} \\
                    14\pi &= 8\pi \cdot 3 \cdot \frac{dr}{dt} \\
                    \frac{14\pi}{3\cdot 8\pi} = \frac{dr}{dt} \\
                    \frac{7}{12} &= \frac{dr}{dt} 
                \end{aligned}    
            \]

            Lastly, solve for $\frac{dV}{dt}$ using the above info

            \[
                \begin{aligned}
                    V &= \frac{4}{3}\pi r^3 \\
                    \frac{dV}{dt} &= 4\pi r^2 \cdot \frac{dr}{dt} \\
                    \frac{dV}{dt} &= 4\pi \cdot 3^2 \cdot \frac{7}{12} \\
                    \frac{dV}{dt} &= 21\pi
                \end{aligned}    
            \]

        \end{enumerate}

        \subsection{Local Linearity and Approximation}

        \textbf{Local linearity} is the idea that if we zoome in really close to a point on a graph that is differeitable at all points
        in its doman, it would eventually be a straight line, \textit{a tangent line}.

        The general formula for the tangent line at a point $x = a$ is:

        \[
            y = u^{\prime}(a)(x - a) + u(a)  
        \]

        \subsection{L'Hopitals Rule}

        \noindent L'Hopitals Rule states that:

        \begin{gather*}
            \text{IF} \\
            \lim_{x \to c} \frac{f(x)}{g(x)} = \frac{0}{0} \quad \text{OR} \quad \lim_{x \to c} \frac{f(x)}{g(x)} = \frac{\pm\infty}{\pm\infty}\\
            \text{THEN} \\
            \lim_{x \to c} \frac{f(x)}{g(x)} = \lim_{x \to c} \frac{f^{\prime}(x)}{g^{\prime}(x)}
        \end{gather*}

        \section{Analytical Applications of Differentiation}

        \subsection{Mean Value Theorem}

        The mean value theorem states that if a function is continous on a closed interval $[a, b]$ and differentiable on the 
        open interval $(a, b)$, then there is at least one point $c$ in the open interval such that: 

        \[
            f^{\prime}(c) = \frac{f(b) - f(a)}{b - a}    
        \]

        \noindent \textbf{\textit{Extra Cool Stuff (Notation)}}

        They written above is the correct way taught in the AP curriculum, but you can also shorten it as follows:

        The mean value theorems states that $f \in C[a, b] \cap C^1(a, b) \to \exists c\in(a, b)$ such that: 

        \[
            f^{\prime}(c) = \frac{f(b) - f(a)}{b - a}    
        \]

        \textbf{Warning: }\textit{Although this may seem cool, I would not reccomend it as it is confusing and VERY easy to mess up, stick with the normal notation}

        \subsection{Extreme Value Theorem}

        The extreme value theorem states that if a function is continous over a closed invterval $[a, b]$, then it has both a 
        maximum and minimum value on that interval. There must also exist $c$ and $d$ in $[a, b]$ such that:

        \[
            f(c) \le f(x) \le f(d) \quad \forall x \in [a, b]    
        \]

        \section{Intergration and Accumulation of Change}

        The accumulation of change is the net change of a quantity. This is not the same
        thing as simply the quantity. The accumulation of change takes into consideration time, 
        and it is the quantity over a specified time period. There may be some quantity before or 
        after this time period, but we would not count that. 

        \subsection{Riemann Sums}

        A Riemann sum is a way to approximate the area under a curve. It splits up the area into several
        rectangles, where the heights match the function. The sum of the area provides a decent
        approximation of the area under the curve.

        \subsubsection{Type of Riemann Sums}

        \begin{itemize}
            
            \item \textbf{Left Riemann Sums} line up the left side of the rectangle with the curve of the function
            \item \textbf{Right Riemann Sums} line up the right side of the rectangle with the curve of the function
            \item \textbf{Midpoint Riemann Sums} line up the midpoint of the rectangle with the curve of the function
            \item \textbf{Trapezoidal Riemann Sums} line up the trapezoids with the curve of the function

        \end{itemize}

        With each type of sum, the more rectangles used, the more accurate the approximation. This can be achieved my making 
        $\Delta x$, the base of the rectangle, smaller and smaller. 

        \subsubsection{Riemann Sums in Summation Notation}

        Given any Riemann sum over an interval $[a, b]$ with $n$ rectangles of \textit{equal width}, the sum can be written as,
        $\Delta x$ can be defined as $\frac{b - a}{n}$. The bottom right corner of each rectangle will be $x_i$, so $x_i = a + \Delta \cdot i$

            \begin{center}
                \begin{tabular}{ |  c  |  c  |}
                    \hline
                    Left & Right \\
                    \hline
                    $\sum_{i = 0}^{n - 1} \Delta x \cdot f(x_i)$ & $\sum_{i = 1}^{n} \Delta x \cdot f(x_i)$ \\
                    \hline
                \end{tabular}
            \end{center}

        \subsection{Definite Integrals}

        As $\Delta x$ gets smaller and smaller in the Riemann Sums, it is possible to get better approximations of the area under a curve.
        However, it is imposssible to calculate the exact area under a cruev using a finite number of rectangles. In order to do 
        so, an infinitely small $\Delta x$ would be needed. Definite integrals can find the \textit{exact} area of an interval under
        a curve. 

        The definite integral notation over $[a, b]$ of $f(x)$ is:

        \[
            \int_a^b f(x) \space dx
        \]
        
        The connection between definite integrals and Riemann Sums is as follows:

        \[
            \int_a^b f(x) \space dx = \lim_{n \to \infty} \sum_{i = 1}^{n} \Delta x \cdot f(x_i)    
        \]

        where $\Delta x = \frac{b - a}{n}$ and $x_i = a + \Delta x \cdot i$

        \subsubsection{Properties of Definite Integrals}

        The properties of definite integrals are relatively similar to those of derivatives, although,
        I would be careful when doing them to not mix them up. They are as follows:

        \begin{enumerate}
            
            \item 
            \[
                \int_a^a f(x) dx = 0    
            \]

            \item 

            \[
                \int_a^b f(x) dx = -\int_b^a f(x) dx    
            \]

            \item 

            \[
                \int_a^b kf(x) dx = k\int_a^b f(x) dx    
            \]

            \item 
            \[
                \int_a^b \left[f(x) \pm g(x)\right] = \int_a^b f(x) dx \pm \int_a^b g(x) dx
            \]

            \item
            \[
               \int_a^b f(x) dx = \int_a^c f(x)dx + \int_c^b f(x)dx
            \]

        \end{enumerate}

        \subsection{The Fundamental Theorem of Calculus}

        The fundamental theorem of calculus states:

        Let $f$ be a continous function over $[a, b]$, and let:

        \[
            F(x) = \int_a^x f(t) \space dt    
        \]

        $F$ is the antiderivative of $f$, or in other words, the derivative of $F$ is $f$. This can 
        explain how to solve for the area under a curve using antidiffferentiation.

        \[
            \int_a^b f(x) \space dx = F(b) - F(a)    
        \]

        \noindent \textit{Note:} The notation is the same as the one above and is commonly used

        \[
            \int_a^b f(x) \space dx = F(x) {|}_a^b   
        \]

        \subsection{Indefinite Integrals}

        Indefinite integrals, compared to their definite counterparts, have to no bounds
        and the symbol $\int$ is used to find the antiderivative of a function. 

        When differentiating, all constants are lost and so there are an infinte possibility
        of constants in the antiderivative. To account for this, a $+C$ is added.

        \[
            F(x) + C = \int f(x)dx    
        \]

        The above is true where $F(x) = f^{\prime}(x)$

\end{document}