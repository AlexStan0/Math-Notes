\documentclass[11pt]{article}
\usepackage{amsmath,amssymb,amsthm,enumerate,nicefrac,fancyhdr,hyperref,graphicx,adjustbox}
\hypersetup{colorlinks=true,urlcolor=blue,citecolor=blue,linkcolor=blue}
\usepackage[left=2.6cm, right=2.6cm, top=1.5cm, includehead, includefoot]{geometry}
\usepackage[dvipsnames]{xcolor}
\usepackage[d]{esvect}

%% commands
%% useful macros [add to them as needed]
% sets
\newcommand{\C}{{\mathbb{C}}} 
\newcommand{\N}{{\mathbb{N}}}
\newcommand{\Q}{{\mathbb{Q}}}
\newcommand{\R}{{\mathbb{R}}}
\newcommand{\Z}{{\mathbb{Z}}}
\newcommand{\F}{{\mathbb{F}}}

% bases
\newcommand{\mA}{\mathcal{A}}
\newcommand{\mB}{\mathcal{B}}
\newcommand{\mC}{\mathcal{C}}
\newcommand{\mD}{\mathcal{D}}
\newcommand{\mE}{\mathcal{E}}
\newcommand{\mL}{\mathcal{L}}
\newcommand{\mM}{\mathcal{M}}
\newcommand{\mO}{\mathcal{O}}
\newcommand{\mP}{\mathcal{P}}
\newcommand{\mS}{\mathcal{S}}
\newcommand{\mT}{\mathcal{T}}

% linear algebra
\newcommand{\diag}{\operatorname{diag}}
\newcommand{\adj}{\operatorname{adj}}
\newcommand{\rank}{\operatorname{rank}}
\newcommand{\spn}{\operatorname{Span}}
\newcommand{\proj}{\operatorname{proj}}
\newcommand{\prp}{\operatorname{perp}}
\newcommand{\refl}{\operatorname{refl}}
\newcommand{\tr}{\operatorname{tr}}
\newcommand{\nul}{\operatorname{Null}}
\newcommand{\nully}{\operatorname{nullity}}
\newcommand{\range}{\operatorname{Range}}
\renewcommand{\ker}{\operatorname{Ker}}
\newcommand{\col}{\operatorname{Col}}
\newcommand{\row}{\operatorname{Row}}
\newcommand{\cof}{\operatorname{cof}}
\newcommand{\Num}{\operatorname{Num}}
\newcommand{\Id}{\operatorname{Id}}
\newcommand{\ipb}{\langle \thinspace, \rangle}
\newcommand{\ip}[2]{\left\langle #1, #2\right\rangle} % inner products
\newcommand{\M}[2]{M_{#1\times #2}(\F)}
\newcommand{\RREF}{\operatorname{RREF}}
\newcommand{\cv}[1]{\begin{bmatrix} #1 \end{bmatrix}}
\newenvironment{amatrix}[1]{\left[\begin{array}{@{}*{\numexpr#1-1}{c}|c@{}}}{\end{array}\right]}
\newcommand{\am}[2]{\begin{amatrix}{#1} #2 \end{amatrix}}

% vectors
\newcommand{\vzero}{\vv{0}}
\newcommand{\va}{\vv{a}}
\newcommand{\vb}{\vv{b}}
\newcommand{\vc}{\vv{c}}
\newcommand{\vd}{\vv{d}}
\newcommand{\ve}{\vv{e}}
\newcommand{\vf}{\vv{f}}
\newcommand{\vg}{\vv{g}}
\newcommand{\vh}{\vv{h}}
\newcommand{\vl}{\vv{\ell}}
\newcommand{\vm}{\vv{m}}
\newcommand{\vn}{\vv{n}}
\newcommand{\vp}{\vv{p}}
\newcommand{\vq}{\vv{q}}
\newcommand{\vr}{\vv{r}}
\newcommand{\vs}{\vv{s}}
\newcommand{\vt}{\vv{t}}
\newcommand{\vu}{\vv{u}}
\newcommand{\vvv}{{\vv{v}}}
\newcommand{\vw}{\vv{w}}
\newcommand{\vx}{\vv{x}}
\newcommand{\vy}{\vv{y}}
\newcommand{\vz}{\vv{z}}

% display
\newcommand{\ds}{\displaystyle}
\newcommand{\qand}{\quad\text{and}}
\newcommand{\qandq}{\quad\text{and}\quad}
\newcommand{\hint}{\textbf{Hint: }}

% misc
\newcommand{\area}{\operatorname{area}}
\newcommand{\vol}{\operatorname{vol}}
\newcommand{\red}[1]{{\color{red} #1}}
\newcommand{\rc}{\red{\checkmark}}

%% header
\pagestyle{fancy}
\fancyhead[L]{\bf\large AP Calc AB \\ Presentation Question Solutions}
\fancyhead[R]{\bf\large Spring 2024 \\}
\setlength{\headheight}{35pt}

\begin{document}

	\begin{enumerate}[{\bf S1.}] 

        \item $(\Z, +)$ - The integers under addition is a group, here is how it satisfies the axioms of a group
        
            \begin{enumerate}
                
                \item
                
                    Given any two integers $x, y\in\Z$, their sum, $x + y$ is bound to be within
                    the set of integers. Therefore the set $\Z$ is clsoed under addition

                \item 

                    $0$ is the identity element, as, given $e = 0$, $x + e = e + x = x$ where $e = 0$ holds true. 

                \item 

                    For every $x\in\Z$ there exists $y\in\Z$ such that $x + y = y + x = 0$. In this specific instance
                    $y = -x$. Therefore, each element in the set has an inverse within the set. 

                \item 

                    Given any $x, y, z\in\Z$. The assoicative statement $x + (y + z) = z + (x + y)$ holds true.

            \end{enumerate}

            Therefore, as all of the axioms of a group are met, $(\Z, +)$, the set of integers 
            under addition, is a group. \\

        \item $(\Q, \times)$ - The rations under multiplication, is this a Group?
        
            \begin{enumerate}
                
                \item 

                    Given any two $x, y\in\Q$, the product of their multiplication $x\times y$ will always result in a 
                    rational number. Therefore, the set $\Q$ is closed under multiplication

                \item 

                    The identity element of $\Q$ under multiplication is $1$ as, for all $x\in\Q$, where $e = 1$, the 
                    identity $x\times e = e\times x = x$ holds true. 

                \item 

                    Multiplication is associative, same as addition, therefore, for any $x, y, z\in\Q$, the equation
                    $x\times(y\times z) = z\times(x\times y)$ is true.

                \item 

                    Now, $(\Q, \times)$ fails to meet the criteria of a group when it comes to inverses. For an element $x\in\Q$
                    to have an inverse under multiplication there must exist $y\in\Q$ such that $x\times y = y\times x = 1$. \\

                    As the set of rational integers $\Q$ includes the element $0$, which does not have an inverse, it means 
                    that the inverse axiom, stating that each element in the set must have an inverse such that $x\times y = y\times x = 1$
                    is not met. Therefore, $\Q$ is not a group under multiplication

            \end{enumerate}

        \newpage

        \item $(\Q, +, \times)$ is the ring of rationals closed under addition and multiplication, how does it satisfy the axioms of a ring?
        
            \begin{enumerate}
                
                \item 

                    It is a group under addition, meaning

                    \begin{itemize}
                        \item \textbf{Closure:} For all $x, y\in\Q$, $x + y \in\Q$ holds true
                        \item \textbf{Identity:} For any $x\in\Q$ given $e = 0$, $x + e = e + x = x$ holds true
                        \item \textbf{Inverse:} For any $x\in\Q$ there exists $y\in\Q$ such that $y = -x$ and $x + y = y + x = e$ where $e = 0$
                        \item \textbf{Associativity: } As addition is associative, for all $x, y, z\in\Q$, the equation \\
                        $x + (y + z) = z + (x + y)$ holds true
                    \end{itemize}

                \item 

                    When it comes to the associativity of $\times$, for every $x, y, z\in\Q$, $x\times(y\times z) = z\times(x\times y)$ holds true.

                \item 

                    For the distributive properties, it is the exact same scenario. Given $x, y, z\in\Q$, the following properties

                    \[
                        x \times (y + z) = (x\times y) + (x\times z)
                    \]

                    and 

                    \[
                        (y + z) \times x = y\times x + z\times x
                    \]

                    Both hold true, meaning that the axiom of distributive properties holds true in this case. 

            \end{enumerate}

            Therefore, $(\Q, +, \times)$ is a ring. 

        \item Using what was just learnt, decide whether $(\mathbb{O}^+, +, \times)$, the set of positive odd integers, 
              is a ring. Why or why not?

              \begin{enumerate}
                
                \item 

                    This set under addition and multiplication is not a ring as the axiom of being a group under addition is not met, here
                    is why:

                    \begin{itemize}
                        
                        \item To start, for any $x, y\in\mathbb{O}^+$, their product $x + y$, will not always be in the set 
                              of odd positive integers as, for example, $1 + 3 = 4$. Therefore, the set $\mathbb{O}^+$ is not closed under
                              addition

                        \item Given $x\in\mathbb{O}^+$ there must exist an identity element $e$ such that $x + e = e + x = x$. 
                              Although $0$ is usually the identity element under addition, as $0\notin\mathbb{O}^+$, there is no element in the
                              set that can satisfy this axiom.

                        \item For any $x\in\mathbb{O}^+$ there must exist $y\in\mathbb{O}^+$ such that $x + y = y + x = 0$. But, as 
                              the set $\mathbb{O}^+$ does not include any negative numbers, no elements within the set have an inverse
                              within it. 


                    \end{itemize}

              \end{enumerate}

        \newpage

        \item Is the set of rational numbers $\Q$ a field?
        
              \begin{enumerate}
                
                \item As discussed in S2, $\Q$ is a valid group under addition, therefore the first axiom is satisfied.
                
                \item This is where it gets interesting. The second axiom to define a field states that a set without the identity set 
                      $0$ (i.e $\Q - \{0\}$) must be a valid group under multiplication. We know that $\Q$ satisfies all of the 
                      conditions of a group under multiplication, save for the fact that $0$ does not have inverse within the set.

                      Therefore, as $0$ is no longer within the set, for any $x\in\Q$ there exists $y\in\Q$ such that $x\times y = y\times x = e$
                      where $e = 1$                      

              \end{enumerate}

              Therefore, as both field axioms are obeyed, the set $\Q$ is a field. 

        \item Is the set of complex numbers $\C$ a field?
        
        \begin{enumerate}
            
            \item $(\C, +)$ is a valid group 
            
            \begin{itemize}
                
                \item For all $x, y\in\C$, $x + y\in\C$ holds true
                \item For any $x\in\C$, $x + 0 = 0 + x = x$, therefore the identity element is $0$
                \item For all $x\in\C$, there exists $y\in\C$ such that $x + y = y + x = 0$
                \item For all $x, y, z\in\C$, $x + (y + z) = z + (x + y)$ is true

            \end{itemize}

            \item $(\mathbb{S}, \times)$ is also a valid group where $\mathbb{S} = \C - \{0\}$
            
            \begin{itemize}
                
                \item For all $x, y\in\mathbb{S}$, $x\times y\in\mathbb{S}$ holds true
                \item Given any $x\in\mathbb{S}$, $x\times 1 = 1 \times x = x$, therefore the identity element is $1$
                \item For any $x\in\mathbb{S}$ there exists $y\in\mathbb{S}$ such that $x\times y = y\times x = e$ where $e$ is the 
                      identity element $1$
                \item For all $x, y, z\in\mathbb{S}$, $x\times(y\times z) = z\times (x\times y)$

            \end{itemize}

            Therefore, as all of the field axioms are satisfied, $\C$ is a field. 

        \end{enumerate}

    \end{enumerate}

\end{document}