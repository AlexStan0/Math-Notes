\documentclass{article}

\usepackage[a4paper, margin=1in]{geometry}
\usepackage{amsmath}
\usepackage{fancyhdr}
\usepackage{graphicx}

\setlength{\parskip}{1.5em}
\newcommand{\C}{\mathal{C}}
\newcommand{\qspace}{\vspace*{1.7em}}

\pagestyle{fancy}
\fancyhead[L]{\bf\large IDC4UP: AP Calculus AB \\ FRQ Assignment}
\fancyhead[R]{\bf\large March 2024 \\}
\setlength{\headheight}{35pt}

\title{
    \textbf{AP Calculus AB: \\Free-Response Question}
    \author{Alexandru Stan}
    \date{March 10, 2024}
}

\begin{document}

    \maketitle
    \vfill
    \newpage

    \section*{Introduction}

    A Calculator is not allowed for this question, and, all AP guidlines (as stated below) 
    are to be followed. 

    Show all of your work, even though the question may not explicitly remind you to do so. Clearly label any
    functions, graphs, tables, or other objects that you use. Justifications require that you give mathematical reasons,
    and that you verify the needed conditions under which relevant theorems, properties, definitions, or tests are
    applied. Your work will be scored on the correctness and completeness of your methods as well as your answers.
    Answers without supporting work will usually not receive credit.

    Unless otherwise specified, answers (numeric or algebraic) need not be simplified. If your answer is given as a
    decimal approximation, it should be correct to three places after the decimal point.

    Unless otherwise specified, the domain of a function $f$ is assumed to be the set of all real numbers $x$ for which
    $f(x)$ is a real number.

    \section*{Questions}

    \begin{table}[h]
        \def\arraystretch{2}
        \begin{tabular}{|c|c|c|c|c|c|}
            \hline
            $t_{\text{(hours)}}$ & 1 & 3 & 5 & 6 & 9 \\
            \hline
            $S(t)_{\text{(sales per hour)}}$ & 2 & 11 & 8 & 5 & 2 \\
            \hline 
        \end{tabular}
    \end{table} 

    The rate at which cars are sold at a dealership is modeled by $S(t)$ where $S(t)$ is continous, twice-differentiable function.
    $S$ is measured in sales per hour and $t$ is measured in hours where $0 \le t \le 11$ such that
    $t = 0$ is 8am. Values are given in the above table for selected values of $t$.

    \begin{enumerate}
        
        \item[(a)] Use the data in the table above to approximate $S^{\prime}(4)$. Show the computations that lead to your answer
                   and indicate units of measure. \textbf{[3 pts]} \qspace 
        \item[(b)] Use a right Riemann sum with the four subinvtervals indicated by the data in the table to 
                   approximate $\int_3^6 R(t)dt$. Indicate units of measure. \textbf{[2 pts]} \qspace
        \item[(c)] Is the approximation in part (b) an overestimate or underestimate of $\int_3^6$? \textbf{[2 pts]} \qspace
        \item[(d)] The sum $\sum_{k = 1}^n S\left(3 + \frac{2(k - 1)}{n}\right)\frac{2}{n}$ is a left Riemann sum with $n$ subintervals of equal length. 
                   The limit of this sum as $n$ goes to $\infty$ can be stated as a definite integrals. Express the limit as a definite integral. \textbf{[2 pts]}

    \end{enumerate}
    \newpage


    \fancyhead[L]{\bf\large IDC4UP: AP Calculus AB \\ FRQ Assignment Solutions}
    \fancyhead[R]{\bf\large March 2024 \\}

    \section*{Solutions}

    \begin{enumerate}

		% (a)
		\item[(a)]
		
            \textbf{[1 pt]} To receive the first point, the student must understand that the Mean Value Theorem
            is required. They must also state it out in a fashion that is similar to the following: 

            \begin{center}
                \textit{The Mean Value Theorem states that if a function is continuous on the closed interval $[a, b]$ 
                and and differentiable on the open interval $(a, b)$, then there exists a number $c$ in the 
                open interval $(a, b)$ such that $f^{\prime}(c) = \frac{f(b) - f(a)}{b - a}$.}
            \end{center}

            \textbf{[1 pt]} Another point is then awarded if the student is able to recognize the correct values to plug in the correct values,
            to solve for $S^{\prime}(4)$, which, in this case, are $a = 3$ and $b = 5$.

            \textbf{[1pt]} The student is awarded a point for returning the correct answer as follows below

            \[
                \begin{aligned}
                    S^{\prime}(4) &= \frac{f(5) - f(3)}{5 - 3} \\
                    S^{\prime}(4) &= \frac{8 - 11}{2} \\
                    S^{\prime}(4) &= -\frac{3}{2}
                \end{aligned}    
            \]

            And for stating the correct units, in this case, \textit{sales per hour per hour} or $\frac{\text{sales}}{h^2}$ \qspace
		
		% (b)
		\item[(b)]
		
                \textbf{[1pt]} The student is awarded a point for recognizing the right Riemann Sum and plugging in the correct values
                as shown below

                \[
                        (6)(5) + (5)(8)
                \]

                \textbf{[1pt]} Another point is then awarded for returning a correct answer of $70$, and for stating the 
                correct units of measurement, in this case, \textit{sales} \qspace
		
        \item[(c)]
        
                \textbf{[1pt]} A point is awarded for recognising and stating that in the interval $[3, 6]$ the function is 
                decreasing and concave down.

                \textbf{[1 pt]} Another point is awarded for stating that due to the above reasons, the approximation from the 
                right Riemann sum is an overestimate of the actual value of the integral. \qspace
        
        \item[(d)] 
        
                \textbf{[1 pt]} The student is awarded a point of recognising the correct upper and lower bounds (where $a$ is the lower bound and $b$ is the upper bound) 
                of the integral, in this case, $a=3$ and $b=5$

                \textbf{[1 pt]} Another point is awarded for the student correctly recognizing the intergrand and correctly
                stating all the above factors in the form of a definite integral, as shown below

                \[
                    \int_3^5 S(t)dt    
                \]
		
    \end{enumerate}

\end{document}