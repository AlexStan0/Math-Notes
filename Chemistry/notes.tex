\documentclass[12pt, a4paper]{article}

\usepackage[utf8]{inputenc}
\usepackage[english]{babel}
\usepackage{indentfirst}
\usepackage{graphicx}
\usepackage{caption}
\usepackage{float}
\usepackage[margin=1.2in]{geometry} % margins
\usepackage{multicol}
\usepackage{wrapfig}
\usepackage[colorlinks=true, linkcolor=blue, urlcolor=cyan]{hyperref}
\usepackage{fancyhdr}
\usepackage{chemfig}
\usepackage[version=4]{mhchem}
\usepackage{tikz}
\usepackage{amsmath}

\hypersetup {
    linkcolor=black
}

\renewcommand{\baselinestretch}{1.5}
\newcommand{\fline}{\par\noindent\rule{\textwidth}{0.1pt}}
\newcommand{\C}{\ce{C}}
\setlength{\parskip}{1em}
\addto\captionsenglish{\renewcommand{\contentsname}{Units}}

% Create commands for the molecule states
\newcommand{\gas}{_{\text{(g)}}}
\newcommand{\solid}{_{\text{(s)}}}
\newcommand{\liquid}{_{\text{(l)}}}

\pagestyle{fancy}

\fancyhead[L]{\bf\large 4U Chemistry - Notes}
\fancyhead[R]{\bf\large Semester 2}
\setlength{\headheight}{35pt}

% Create title
\title{
    \textbf{4U Chemistry:\\ Notes, Drawings, Examples}
    \author{Alexandru Stan}
    \date{Start date: February 2, 2024 \\ End date: June 27th, 2024}
}

\begin{document}

    \maketitle
    \vfill
    \newpage 

    \tableofcontents
    \fline
    \newpage

    \section{Organic Chemistry}

    \subsection{Reactions of Hydrocarbons}

    There are serveral reactions of hydrocarbons (\~20 I think?) and this is
    too low of a level to logic it out, so it will all be memorization, get
    ready.

    \subsection{Alkane Reactions}

    In general, alkanes are fairly unreactive, however, the burn very easily in combustion
    reactions, releasing a lot of energy.

    Reactants of the reaction will be the hydrocarbon and $\ce{O_{2\text{(g)}}}$ with the products
    of a complete combustion being $\ce{CO_{2\text{(g)}} + H2O_{\text{(g)}}}$

    \subsubsection{Alkane Reactions - Substitution Reaction}

    During a substitution replaced a $\ce{H}$ atom with a halogen to make
    an alkyl halide. This reaction only occurs with three halogen gases $\ce{F2}$, 
    $\ce{Cl2}$, and $\ce{Br2}$. 

    This reaction can create a mixture of different isomers and can occur in multiple
    phases, making multiple substituions if needed.

    \textbf{\textit{NOTE:}} This reaction required UV light to occur

    \begin{center}
        \chemfig{C(-[:90]H)(-[:180]H)(-[:270]H) - C(-[:90]H)(-[:0]H)(-[:270]H)}
        + $\ce{Br_{2\gas} ->[\text{UV Light}]}$ \chemfig{C(-[:90]H)(-[:180]H)(-[:270]H) - C(-[:90]H)(-[:0]Br)(-[:270]H)}
        + $\ce{HBr_{\gas}}$
    \end{center}
    
    \subsection{Alcohols}

    An alcohol group is an organic compound that contains the hydroxyl $\ce{-OH}$ functional group
    the ``alcohol'' in beer and wine is really ``ethanol''

    \begin{center}
        \begin{tabular}{| c | c | c | c |}
            \hline
            $\ce{CH3OH}$ & $\ce{CH3CH2OH}$ & $\ce{CH3CH2CH2OH}$ & $\ce{CH3CH2CH2CH2OH}$\\
            \hline
            methanol & ethanol & propan-1-ol & butan-1-ol \\
            \hline
        \end{tabular}
    \end{center}

    \noindent\textbf{Some Important Alcohols:}
    \begin{itemize}
        \item Methanl - prodiced from wood, often used a solvent but is toxic
        \item Isopropanol (propan-2-yl) - Rubbing alcohol, used as an antiseptic
        \item Glycerol - Used to make fats in the body. (propan-1,2,3-ol)
    \end{itemize}

    \subsection{Naming Alcohols}

    When naming alcohol, the following rules need to be followed in this exact order:

    \begin{center}
        \begin{enumerate}
            \item Identify the longest chain of $\ce{C}$ that contains the $\ce{-OH}$ (hydroxyl) group
            \item Number the $\ce{C}$ atoms with the \#$1$ closest to the $\ce{-OH}$. It has prority over the alkyl 
                  groups and halogens
            \item Drop the -e ending (if two vowels are present) on alkane and add -ol. Use a number
                  if needed before the -ol
            \item Name any side branches % TODO: double check this!!
    
        \end{enumerate}
    \end{center}

    \noindent Examples:

    \begin{itemize}
        
        \item pentan-1-ol 
        
        \[
            \ce{CH3-CH2-CH2-CH2-CH2-OH}    
        \]

        \item pentan-3-ol
        
        \[
            \chemfig{CH3-CH2-CH2(-[6]OH)-CH2-CH3}    
        \]

        \item 

    \end{itemize}

    \subsubsection{Aromatic Alcohols}

    The simples aromatic alcohol is a benzene ring with one hydroxyl group group
    bondede to it. Its IUPAC name is phenol

    \begin{center}
        \chemfig{*6(-=-=(-OH)-=)}
    \end{center}

    If the benzene ring has two $\ce{-OH}$ groups attached, the name is based on benzene
    and inlcudes numbers for the $\ce{-OH}$ groups

    \begin{center}
        \chemfig{*6(-=(-OH)-(-OH)=-=)}
    \end{center}

    \subsection{Primary, Secondary, and Tertiary Alchols}

    Alcohols are classified acording to where the $\ce{-OH}$ is attached. They
    are classified as primary, secondary, and tertiary alchols as stated below:

    \begin{itemize}
        \item $1^{\circ}$ (Primary Alcohol) - Hydroxyl group attached to the end 
        \item $2^{\circ}$ (Secondary Alcohol) - Hydroxyl group attached to a $\ce{C}$ that is attached
                          to two other $\ce{C}$% TODO: finish this section
        \item $3^{\circ}$ (Tertiary Alchol) - Hydroxyl group attached to a $\C$ that is attached to
                          three other $\C$
    \end{itemize}

    \subsection{Polyalcohols}

    Polyalcohols are just alcohols with more than one hydroxyl. For Nomencalture, use suffixes
    (di, tri, etc.) and numbers. The -e is kept if it followed by a consonant but is dropped
    if followed by a vowel

    \subsection{Properties of Alcohols}

    The presence of $\ce{-OH}$ group makes the molecule polar that can form hydrogen bonds. The longer
    the Carbon ($\C$) the less polar it is. Small alcohols are completely soluble in water, but the 
    solubility decreases as the length of the carbon chain increases. 

    \subsubsection{BP and MP}

    Alcohols can hydrogen bond and have higer MP and BP than 
    hydrocarbons of similar sizes

    \begin{center}
        \begin{tabular}{ | c | c | c |}
            \hline
            Molecule & Molar Mass & Boiling Point \\
            \hline
            Propane & $44$ g/mol & $-42.1^{\circ}$C \\
            \hline
            Ethanol & $46$ g/mol & $78.3^{\circ}$C \\
            \hline
        \end{tabular}
    \end{center}

    \subsection{Ethers}

    In an ether, the functional group consists of two C atoms connected to a single $\ce{O}$
    atom. The $\ce{C - O}$ bond is polar and the shape is bent, making it a polar molecule. Ethers
    are also known to be good solvents

    \begin{center}
        \chemfig{C(-[:90])(-[:180])(-[:270])-\charge{90=\:, 270=\:}{O}-C(-[:90])(-[:0])(-[:270])}\\
        \vspace*{2mm}
        An \textbf{ether}
    \end{center}

    \noindent\textbf{Some Important Ethers}

    \begin{itemize}
        
        \item Ethylene Oxide - Used in Epoxy
        \item MBTE - A gasoline aditive that helps gasoline burn better
        \item 18-crown-6 - Built by Pedersen and Cram (Nobel Prize 1987, capable of building metal ions)

    \end{itemize}

    \subsection Naming Ethers
    
    When naming ethers you need to follow a set or rules are you do alcohols and other substances
    , although, keep in mind that ethers are made up of two parts. The rules
    are as follows

    \begin{enumerate}
        
        \item Name the smallest alkyl group 1st, drop the -yl, and add oxy. The will
              be named as a side chain
        \item End with the longest alkyl group last named as an alkane. This is the parent chain

    \end{enumerate}

\end{document}