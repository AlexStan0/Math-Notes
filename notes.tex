\documentclass[14pt]{article}

% Import dependencies
\usepackage[a4paper]{geometry}
\usepackage[utf8]{inputenc}
\usepackage[myheadings]{fullpage}
\usepackage{fancyhdr}
\usepackage{lastpage}
\usepackage{float}
\usepackage{graphicx, wrapfig, subcaption, setspace, booktabs}
\usepackage{graphicx}
\usepackage[T1]{fontenc}
\usepackage[font=small, labelfont=bf]{caption}
\usepackage{fourier}
\usepackage[protrusion=true, expansion=true]{microtype}
\usepackage[english]{babel}
\usepackage{sectsty}
\usepackage{url, lipsum}
\usepackage[T1]{fontenc}
\usepackage{icomma}
\usepackage{siunitx}
\usepackage{ragged2e}
\usepackage{amsmath}
\usepackage{comment}
\usepackage{tikz}
\usepackage[bottom]{footmisc}
\usepackage{hyperref}
\usepackage[framemethod=TikZ]{mdframed}

% Create a new command for horizontal lines without indentation
\newcommand{\HRule}[1]{\rule{\linewidth}{#1}}
\onehalfspacing
\setcounter{tocdepth}{5}
\setcounter{secnumdepth}{5}

\hypersetup{
    colorlinks,
    citecolor=black,
    filecolor=black,
    linkcolor=black,
    urlcolor=black
}

% Change font-family to a monospace font
\renewcommand{\familydefault}{\rmdefault}

% Remove the line above footnotes
\let\footnoterule\relax

% Set the page numbers to be arabic (i.e normal numbers)
\pagenumbering{arabic}

%-------------------------------------------------------------------------------
% HEADER & FOOTER
%-------------------------------------------------------------------------------
\pagestyle{fancy}
\fancyhf{}
\setlength\headheight{15pt}
\renewcommand{\footrulewidth}{0.4pt}
\fancyhead[L]{Stan \thepage}
\fancyfoot[R]{\thepage\space\text{of}\space\pageref{LastPage}}

\begin{document}

    % Generate the title with the class code & name
    \title{ \normalsize \textsc{MCV4U}
            \\ [2.0cm]
            \HRule{0.5pt} \\
            \LARGE \textbf{\uppercase{Calculus \& Vectors}}
            \HRule{2pt} \\ [0.5cm]
            \normalsize \today \vspace*{5\baselineskip}}

    \date{}

    \author{Alexandru Stan}
    \maketitle

    % Count the page nunbers, and create a new page
    \newpage

    % Set the right header to indicate what is occuring
    \fancyhead[R]{Table of Contents}
    
    % Create the section and sub-sections for the table of contents
   \tableofcontents\newpage
    
    %Sections and subsections for the ToC
    \section{Vectors}


        % Set the right header to indicate what is occuring
        \fancyhead[R]{Vectors}

        \begin{flushleft}

             % Write a simple introduction into what vectors are before we move on
            Vectors are mathematical entities that extend our understanding beyond the 
            one-dimensional quantities. Unlike scalar values that only have magnituide, 
            vectors incorporate both magnitude and direction, offering a versatile toolkit
            for describing dynamic sytstems. 

            % List some scalar and vector quantities
            \begin{mdframed}

                Ex. Scalar Vs. Vector Quantities

                \begin{itemize}
                    \item \textbf{Scalar Quantities:} Mass, Temperature, Time, Distance, Speed, 
                    Energy, Work, Power, etc.
                    \item \textbf{Vector Quantities:} Displacement, Velocity, Acceleration, 
                    Force, Momentum, etc.
                \end{itemize}
            \end{mdframed}

            % State the different ways vectors can be written and the different ways they can be represented visually
            When written in mathematical equations, vectors are usually represented via a a symbol with a vector indicator (i.e $\vec{v}$)
            or via a jointery of the two points (i.e $\vec{AB}$ is a vector from point A to point B) Vectors 
            can also be represented in many other ways, but the most common ways are: algebraically, numerically, 
            and geometrically. Below are examples of each:

            % Create a list of examples of the different ways vectors can be represented
            \begin{itemize}

                \item \textbf{Algebraically:} $\vec{a} = \langle x, y \rangle$
                \item  \textbf{Numerically:} $\vec{a} = \left[ x, y, z \right]$ \footnote{Numerical vectors can also be written as column matrices}
                \item { \textbf{Geometrically:} 

                    \begin{center}

                        % Draw a vector and label the ends as A and B
                        \begin{tikzpicture}
                        
                            % Draw & Label the Vector Arrow
                            \draw[->] (0, 0) -- (6, 2) node[midway, above, sloped] {$\vec{AB}$};

                            % Label the points
                            \node[right] (B) at (6, 2) {B};
                            \node[left] (A) at (0, 0) {A};

                        \end{tikzpicture}

                    \end{center}
                                        
                } 

            \end{itemize}

            % Describe opposite and equal vectors then paint an image of them 
            Vectors can be equal (or equivalent) to each other. For two vectors to be equal (or equivalent) 
            they must have the same magnitude and direction. Vectors can also be opposite to each other; to be
            opposing vectors must have the same magnitude but opposite directions 
            (i.e $\vec{v} = -\vec{v}$).

            % Create two figures to represent equal and opposite vectors
            \begin{figure}[h]
                \begin{subfigure}[b]{0.5\linewidth}
                    \centering
                    \begin{tikzpicture}
                        \draw[->] (0, 0) -- (2, 4) node[midway, above, sloped] {$\vec{AB}$};
                        \draw[->] (3, 0) -- (5, 4) node[midway, above, sloped] {$\vec{CD}$};
                        \node[left] (A) at (0, 0) {A};
                        \node[right] (B) at (2, 4) {B};
                        \node[left] (C) at (3, 0) {C};
                        \node[right] (D) at (5, 4) {D};
                    \end{tikzpicture}%
                    \caption{Equivalent Vectors} \label{fig:eq_vec}
                \end{subfigure}
                %
                \begin{subfigure}[b]{0.5\linewidth}
                    \centering
                    \begin{tikzpicture}
                        \draw[->] (0, 0) -- (2, 4) node[midway, above, sloped] {$\vec{AB}$};
                        \draw[->] (5, 4) -- (3, 0) node[midway, above, sloped] {$\vec{CD}$};
                        \node[left] (A) at (0, 0) {A};
                        \node[right] (B) at (2, 4) {B};
                        \node[right] (C) at (5, 4) {C};
                        \node[left] (D) at (3, 0) {D};
                    \end{tikzpicture}%
                    \caption{Opposite Vectors} \label{fig:op_vec}
                \end{subfigure}
            \end{figure}

            % Force the next section to a new page
            newpage

        \end{flushleft}

        \subsection{Vector Addition and Substraction}

            \begin{flushleft}



            \end{flushleft}

        \subsection{Scalar Multiplication}
        \subsection{Properties of Vectors}
        \subsection{Vectors as Forces}
        \subsection{Vectors as Velocity}
        \subsection{Vectors in R2}
        \subsection{Algebraic Vectors in R3}
        \subsection{Dot Product and Cross Product}
        \subsection{Application of Dot and Cross Product}
        \subsection{Scalar and Vector Projections}

    \section{Lines and Planes}
        \subsection{Vector, Parametric, and Symmetric Equations of a Line}
        \subsection{Vector and Parametric Equations of a Plane}
        \subsection{Cartesian (Scalar) Equation of a Plane}
        \subsection{Intersection of a Lines and Planes}
        \subsection{Intersection of Two Planes}
        \subsection{Intersection of Three Planes}

    \section{Limits and Continuity}
        \subsection{Introduction to Limits}
        \subsection{Special Limits with Trigonometric Functions}
        \subsection{Asymptotes and Holes}
        \subsection{Continuity}

    \section{Derivatives}
        \subsection{Slope of a Curved Line}
        \subsection{The Derivative Function}
        \subsection{Differentiability}
        \subsection{Increasing/Decreasing Functions}
        \subsection{The Chain, Product, and Quotient Rules}
        \subsection{Higher Order Derivatives}

    \section{Curve Sketching}
        \subsection{Points of Inflection}
        \subsection{Curve Sketching Process Given a Function}
    
    \section{Applications of Derivatives}
        \subsection{Velocity and Acceleration}
        \subsection{Optimization With an Equation Given}
        \subsection{Optimization With no Equation loosely dashed-latexGiven}

    \section{Exponential and Trigonometric Functions}
    \subsection{Exponential Functions and Euler's Number}

\end{document}