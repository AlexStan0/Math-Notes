\documentclass[12pt]{article}

% Import dependencies
\usepackage[utf8]{inputenc}
\usepackage[english]{babel}
\usepackage{indentfirst}
\usepackage{graphicx}
\usepackage{caption}
\usepackage{float}
\usepackage[margin=1.2in]{geometry} % margins
\usepackage{multicol}
\usepackage{wrapfig}
\usepackage{amsmath}
\usepackage{amssymb}
\usepackage[colorlinks=true, linkcolor=blue, urlcolor=cyan]{hyperref}

\newcommand{\C}{{\mathbb{C}}} 
\newcommand{\N}{{\mathbb{N}}}
\newcommand{\Q}{{\mathbb{Q}}}
\newcommand{\R}{{\mathbb{R}}}
\newcommand{\Z}{{\mathbb{Z}}}
\newcommand{\F}{{\mathbb{F}}}

\renewcommand{\baselinestretch}{1.5}
\newcommand{\fline}{\par\noindent\rule{\textwidth}{0.1pt}}
\newcommand{\uit}[1]{\textit{#1}}
\setlength{\parskip}{1em}
\addto\captionsenglish{\renewcommand{\contentsname}{Units}}

\hypersetup {
    linkcolor=black
}

% Create title
\title{
    \textbf{AP Calculus AB:\\ Notes, Formulas, Examples}
    \author{Alexandru Stan}
    \date{Start date: February 2, 2024 \\ End date: May 13, 2024}
}

\setlength{\parskip}{1em}

\begin{document}

    \maketitle
    \vfill
    \begin{center}
        \uit{
            Sections based off of Colleged Board units and Mrs. Cooper's Lessons. \\
            Formatting may vary and be of differ in quality
        }
    \end{center}
    \newpage

    \tableofcontents
    \fline
    \newpage

    \section{Limits and Continuity}
    \fline

    The limit is when a given value approaches, or gets \textit{really close} (infinitely) to another value. 
    The standard limit notation is:

    \[
        \lim_{x \to c} f(x)    
    \]

    represents when $x$ can approach $c$ from either left ($-$) or the right ($+$). By adding
    a sign superscript to the $c$, it means that $x$ can only approach from that direction:

    \[
        \lim_{x \to c^+} f(x)    
    \]

    \begin{center}
        \uit{Right hand limit}, $x$ approaches $c$ from values greater than $c$
        \[
            \lim_{x \to c^-} f(x)    
        \]
        \uit{Left hand limit}, $x$ approaches $c$ from values lower than $c$
    \end{center}

    \subsection{Limits to Infinity}

    If a degree (biggest exponent) of a polynomial is greater than or equal to $1$, its
    limit as $x$ approaches $\pm\infty$ will also be $\pm\infty$. This depends on the sign of the leading
    coefficient and the degree of polynomial

    \noindent Example:

    \[
        \begin{aligned}
            f(x) &= 3x^3 - 7x^2 + 2 \\
            \lim_{x \to \infty} f(x) &= \infty \\
            \lim_{x \to -\infty} f(x) &= -\infty     
        \end{aligned}    
    \]

    The degree of $f(x)$ is 3, and the leading coefficient is positive. The graph
    goes down to up from left to right.

    With Fractions, just find whether the highest deree is the numerator or the denominator. 
    Numerator means $\infty$, denominator means $0$

    \subsection{Asympotes}
    Functions can have asymptotes, either vertical or horizontal. In the case of vertical asymptotes, the limit
    would be \uit{unbounded} as it approaches that $x$ value. 

    \noindent Example:

    \[
        \begin{aligned}
            f(x) &= \frac{2x-4}{x-3}\\
            \lim_{x \to 3} f(x) &= \text{undef} \\
            \lim_{x \to 3^-} f(x) &= -\infty \\
            \lim_{x \to 3^+} f(x) &= \infty 
        \end{aligned}
    \]

    As with vertical asymptotes, as $x$ approaches $c$ (in this case $\pm\infty$), the limit would
    approach the horizontal asymptote. Although the $y$-value never actually touches the 
    asymptote, the limit gets really close to the value, from both below and above

    \subsection{Limit Properties}

    The limits of combined functions can be found by finding the limit of each of the 
    individual functions, then applying the operations.

    \begin{itemize}
        
        \item \textbf{Addition/Substraction}
        
        When taking the limit of the sum or difference of multiple functions, it's the 
        same thing as taking the sum of difference of each of the seperate limits of each function
        
        \[
            \lim_{x \to c} [f(x) + g(x)] \implies \lim_{x \to c} f(x) + \lim_{x \to c} g(x)   
        \]

        \[
            \lim_{x \to c} [f(x) - g(x)] \implies \lim_{x \to c} f(x) - \lim_{x \to c} g(x)
        \]

        Note that when the limit of either function is \uit{undefined} the combined
        limit would also be undefined

        \item \textbf{Multiplication}
        
        Multiplication of the limits of functions is quite straightforward
        
        \[
            \lim_{x \to c} [f(x) \cdot g(x)] \implies \lim_{x \to c} f(x) \cdot \lim_{x \to c} g(x)    
        \]

        The same exception applies when one of the limits is \uit{undefined}. This just makes
        the entire combined limit undefined

        \item \textbf{Division}
        
        Division is basically the same as the other basic operations except if the denominator is 0

        \[
            \lim_{x \to c} \frac{f(x)}{g(x)} \implies \frac{\lim_{x \to c} f(x)}{\lim_{x \to c} g(x)}   
        \]

        \item \textbf{Composite Functions}
        
        When working with composite functions, it's the same thing as taking
        the limit of the inner function, then evaluating the outer function normally

        \[
            \lim_{x \to c} f\bigg(g(x)\bigg) \implies f\bigg(\lim_{x \to c} g(x)\bigg)    
        \]

        \item \textbf{Other Theorems}
        
        Given that $\lim f(x)$ and $\lim g(x)$ are both finite for all numbers, and $C$ is a ``constant"

        \[
            \begin{aligned}
                \lim kf(x) &= k \lim f(x) \\
                \lim_{x \to a} C &= C 
            \end{aligned}    
        \]

    \end{itemize}

    \subsection{Solving Limits}

        The first thing to always try to do when solving limits is \textbf{direct substitution}. If this
        is not possible (undefined limit), then algebraic manipulation (factoring) is the next step

        \[
            \begin{aligned}
                & lim_{x \to c} \frac{x^4 + 3x^3 - 10x^2}{x^2 - 2x} \\
                &= \lim_{x \to can} \frac{x^2(x^2 + 3x - 10)}{x(x - 2)} \\
                &= \lim_{x \to c} \frac{x^2(x+5)(x-2)}{x(x-2)}\\
                &= \lim_{x \to c} x^2(x+5)\\
            \end{aligned}    
        \]

        When encountering radicals, conjugates can be used.

        \[
            \begin{aligned}
                & \lim_{x \to c} \frac{x + 4}{\sqrt{3x + 13} - 1} \\
                &= \lim_{x \to c} \frac{x + 4}{\sqrt{3x + 13} - 1} \cdot \frac{\sqrt{3x + 13} + 1}{\sqrt{3x + 13} + 1} \\
                &= \lim_{x \to c} \frac{(x + 4)(\sqrt{3x + 13} + 1)}{3x + 12} \\
                &= \lim_{x \to c} \frac{(x + 4)(\sqrt{3x + 13} + 1)}{3(x + 4)} \\
                &= \lim_{x \to c} \frac{\sqrt{3x + 13} + 1}{3}
            \end{aligned}    
        \]

        When dealing with trigonometric equations, trig identities can be used
        (assuming direct substitution doesn't work)

        \[
            \begin{aligned}
                &\lim_{x \to c} \frac{\cot^2(x)}{1 - \sin(x)} \\
                &= \lim_{x \to c} \frac{\cos^2(x)}{(\sin^2(x))(1 - \sin(x))} \\
                &= \lim_{x \to c} \frac{1 - \sin^2(x)}{(\sin^2(x))(1 - \sin(x))} \\
                &= \lim_{x \to c} \frac{(1 + \sin(x))(1 - \sin(x))}{(\sin^2(x))(1 - \sin(x))} \\
                &= \lim_{x \to c} \frac{1 + \sin(x)}{\sin^2(x)}, \text{for x} \ne (2k + 1)\frac{\pi}{2}
            \end{aligned}    
        \]

        However, functions can not always be factored, so in that case they will just be undefined

        \[
            \begin{aligned}
                & \lim_{x \to 1} \frac{2x}{x^2 - 7x + 6} \\
                &= \lim_{x \to 1} \frac{2x}{(x -6)(x -1)} \\
                &= \frac{2}{0} \\
                &= \text{undef}
            \end{aligned}    
        \]

        \subsection{Continuity}

        A function is continous at a point if its right and left hand side limit at that point are the same. 
        In other words, it can be drown without lifting the pencil.

        \[
            \lim_{x \to c^-} f(x) = \lim_{x \to c^+} f(x) = f(c)    
        \]

        For a function $f$ to be continous for all $\R$ it has to return a real numbe result for all 
        real number values of $x$. Basically $f: \R \to \R$ 

        \begin{itemize}
            
            \item $\sqrt{x + 4}$ is continous $\forall x: x \ge -4$
            \item $\sqrt[5]{x}$ is continous $\forall x: x \in \R$
            \item $\ln x$ is continous $\forall x: x > 0$
            \item $\frac{1}{x-3}$ is continous $\forall x: x \ne 3$

        \end{itemize}

        \textbf{Removable dicontinuity} is function, where a point is ``removed'', and the graph of the new function 
        is almost identical to the original function
        
        \noindent Given:

        \[
            \lim_{x \to c} f(x) = k \le \infty
        \]

        \noindent where:

        \[
            F(x) = \begin{cases}
                f(x) & \text{if } x \ne c \\
                k & \text{if } x = c
            \end{cases}
        \]

        then $F(x)$ has a removable dicontinuity at $k$

        \textbf{Jump dicontinuity} is when the graph jumps from one $y$ value to another at the 
        same $x$-value. 

        \textbf{Infinite dicontinuity} Usually occurs when there is a vertical asymptote, and the 
        dicontinuity occurs over asymptote. Basically, both sides of the asymptote approach that $x$-value, but never actually touch, 
        so the function is not continous. 

        \subsection{Squeeze Theorem}

        When it is to find the limit for a function, the squeeze theorem can be used. Basically, you find
        two other functions, one on top and on below, and use their limits to ``squeeze'' the limit of 
        the given function. 

        \noindent Given:

        \[
            g(x) \le f(x) \le h(x) 
        \]

        \noindent for all $x$ in an open interval that includes $c$, and 

        \[
            \lim_{x \to c} f(x) = \lim_{x \to c} h(x) = L    
        \]
        
        \noindent then,
        \[
            \lim_{x \to c} g(x) = L
        \]
        
        \noindent Note that $x$ and $L$ can both be $\pm\infty$

        \noindent \textbf{Example:}

        \[
            \begin{aligned}
                & \text{Problem:}                          &                                  & \lim_{x \to \infty} \frac{\sin{x}}{x} \\[6pt]
                & \text{keep in mind that}                 & -1                               & \le \sin{x} \le 1                     \\
                & \text{divide by $x$}                     & \frac{-1}{x}                     & \le \frac{\sin{x}}{x} \le \frac{1}{x} \\[6pt]
                & \text{take limits of smaller functions } & \lim_{x \to \infty} \frac{-1}{x} & = 0 = \lim_{x \to \infty} \frac{1}{x} \\[6pt]
                & \text{Squeeze Theorem:}                  & \lim_{x \to \infty}              & \frac{\sin{x}}{x} = 0.
                \end{aligned}    
        \]

        The best way to solve the above problem is to recognmize the easier part of the problem, 
        in this case $\sin x$, then manipulate the inequality in such a way that the middle function becomes the 
        original problem. Solve the limits of the other two functions to solve the original limit

        \subsection{Intermediate Value Theorem (IVT)}

        Given a function $f$ where $f \in C[a, b]$ and $c \in [a, b]$. Then there must a value  $c$ such that $f(a) \le f(c) \le f(b)$. In other words, if a 
        function is continous from $a \to b$, then it must take on every value between $f(a)$ and $f(b)$ for all values of $x$ 
        such that $x \in [a, b]$

        \section{Differentiation: Definition and Fundamental}

        A derivative is the \textbf{instantaneous} rate of change of a function at a point. It's the average
        rate of change over an infintely small interval. It has two main notations. 

        \begin{itemize}
            
            \item \textbf{Lagrange's Notation: } The derivative of $f(x)$ is denoted as $f'(x)$, pronounced as ``f prime of x''. 
                                                 Higher order derivatives are denoted as $f''(x)$ or $f^2(x)$, etc. In general it's 
                                                 written as $f^{n}(x)$, or with the $n$ in ticks.

            \item \textbf{Leibniz's Notation: } The derivative of $f(x)$ is denoted as $\frac{dy}{dx}$, pronounced as ``dee y over dee x''. 
                                                Higher order derivatives are denoted as $\frac{d^2y}{dx^2}$, etc. In general it's 
                                                written as $\frac{d^ny}{dx^n}$

        \end{itemize}

        \subsection{Continuity and Differentiability}

        \noindent\textbf{Differentiability: } A function is differentiable for every value in its domain
        \noindent\textbf{Continuity: } The function has no breaks over its domain, can be drawn without lifting the pencil

        \noindent Differentiability \textit{implies} continuity, but not the other way around

        \subsection{Derivative as a Limit}

        The derivative of a function $f(x)$ at a point $x = a$ is quite truly just first principles, it is as follows:

        \[
            \begin{aligned}
                \frac{d}{dx} f(x) &= \lim_{h \to 0} \frac{f(x + h) - f(x)}{h} \\
                \frac{d}{dx} f(a) &= \lim_{x \to a} \frac{f(x) - f(a)}{h}
            \end{aligned}
        \]

        \subsection{Differentiation Rules}

        \subsubsection{Derivative of a Constant}

        The derivative of a constant is always $0$. This is because the slope of a constant function is always $0$

        \[
            \begin{aligned}
                f(x) &= C \\
                f^{\prime}(x) &= C^{\prime} = 0
            \end{aligned}    
        \]

        \subsubsection{Constant in a function}

        The constant can be moved out in front of the derivative

        \[
            (k f(x))^{\prime} = k f^{\prime}(x)
        \]

        \subsubsection{Sum Rule}

        The derivative of the sum of many functions is the same as the sum of ther derivatives
        of the individual functions. The same applies for subtraction

        \[
            \sum_{k = 1}^n \bigg[f_n(x)\bigg] = \sum_{k = 1}^n \bigg[f_n^{\prime}(x)\bigg]    
        \]

        \subsubsection{Power Rule}

        You put the exponent in front of the function as constant and then substract 1 from the exponent.
        This also applies to negative or fractional exponents (radicals)

        \[
            \begin{aligned}
                f(x) &= x^n : n \in \R \\
                f^{\prime}(x) &= nx^{n-1}
            \end{aligned}    
        \]

        \subsubsection{Product Rule}

        \[
            \bigg[f(x) \cdot g(x)\bigg]^{\prime} = f^{\prime}(x)g(x) + f(x)g^{\prime}(x)    
        \]

        \subsubsection{Quotient Rule} 

        \[
            \bigg[\frac{f(x)}{g(x)}\bigg]^{\prime} = \frac{f^{\prime}(x)g(x) - f(x)g^{\prime}(x)}{g^2(x)}    
        \]

\end{document}