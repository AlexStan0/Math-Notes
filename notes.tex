\documentclass[12pt]{article}

% Import dependencies
\usepackage[utf8]{inputenc}
\usepackage[english]{babel}
\usepackage{indentfirst}
\usepackage{graphicx}
\usepackage{caption}
\usepackage{float}
\usepackage[margin=1.2in]{geometry} % margins
\usepackage{multicol}
\usepackage{wrapfig}
\usepackage{amsmath}
\usepackage{amssymb}
\usepackage[colorlinks=true, linkcolor=blue, urlcolor=cyan]{hyperref}

\renewcommand{\baselinestretch}{1.5}
\newcommand{\fline}{\par\noindent\rule{\textwidth}{0.1pt}}
\newcommand{\uit}[1]{\textit{#1}}
\setlength{\parskip}{1em}
\addto\captionsenglish{\renewcommand{\contentsname}{Units}}

\hypersetup {
    linkcolor=black
}

% Create title
\title{
    \textbf{AP Calculus AP:\\ Notes, Formulas, Examples}
    \author{Alexandru Stan}
    \date{Start date: February 2, 2024 \\ End date: May 13, 2024}
}

\setlength{\parskip}{1em}

\begin{document}

    \maketitle
    \vfill
    \begin{center}
        \uit{
            Sections based off of Colleged Board units and Mrs. Cooper's Lessons. \\
            Formatting may vary and be of differ in quality
        }
    \end{center}
    \newpage

    \tableofcontents
    \fline
    \newpage

    \section{Limits and Continuity}
    \fline

    The limit is when a given value approaches, or gets \textit{really close} (infinitely) to another value. 
    The standard limit notation is:
    \[
        \lim_{x \to c} f(x)    
    \]
    represents when $x$ can approach $c$ from either left ($-$) or the right ($+$). By adding
    a sign superscript to the $c$, it means that $x$ can only approach from that direction:
    \[
        \lim_{x \to c^+} f(x)    
    \]
    \begin{center}
        \uit{Right hand limit}, $x$ approaches $c$ from values greater than $c$
        \[
            \lim_{x \to c^-} f(x)    
        \]
        \uit{Left hand limit}, $x$ approaches $c$ from values lower than $c$
    \end{center}

    \subsection{Limits to Infinity}

    If a degree (biggest exponent) of a polynomial is greater than or equal to $1$, its
    limit as $x$ approaches $\pm\infty$ will also be $\pm\infty$. This depends on the sign of the leading
    coefficient and the degree of polynomial

    \noindent Example:
    \[
        \begin{aligned}
            f(x) &= 3x^3 - 7x^2 + 2 \\
            \lim_{x \to \infty} f(x) &= \infty \\
            \lim_{x \to -\infty} f(x) &= -\infty     
        \end{aligned}    
    \]

    The degree of $f(x)$ is 3, and the leading coefficient is positive. The graph
    goes down to up from left to right.

    With Fractions, just find whether the highest deree is the numerator or the denominator. 
    Numerator means $\infty$, denominator means $0$

    \subsection{Asympotes}
    Functions can have asymptotes, either vertical or horizontal. In the case of vertical asymptotes, the limit
    would be \uit{unbounded} as it approaches that $x$ value. 

    \noindent Example:
    \[
        \begin{aligned}
            f(x) &= \frac{2x-4}{x-3}\\
            \lim_{x \to 3} f(x) &= \text{undef} \\
            \lim_{x \to 3^-} f(x) &= -\infty \\
            \lim_{x \to 3^+} f(x) &= \infty 
        \end{aligned}
    \]

    As with vertical asymptotes, as $x$ approaches $c$ (in this case $\pm\infty$), the limit would
    approach the horizontal asymptote. Although the $y$-value never actually touches the 
    asymptote, the limit gets really close to the value, from both below and above

    \subsection{Limit Properties}

    The limits of combined functions can be found by finding the limit of each of the 
    individual functions, then applying the operations.

    \begin{itemize}
        
        \item \textbf{Addition/Substraction}
        
        When taking the limit of the sum or difference of multiple functions, it's the 
        same thing as taking the sum of difference of each of the seperate limits of each function
        
        \[
            \lim_{x \to c} [f(x) + g(x)] \implies \lim_{x \to c} f(x) + \lim_{x \to c} g(x)   
        \]

        \[
            \lim_{x \to c} [f(x) - g(x)] \implies \lim_{x \to c} f(x) - \lim_{x \to c} g(x)
        \]

        Note that when the limit of either function is \uit{undefined} the combined
        limit would also be undefined

        \item \textbf{Multiplication}
        
        Multiplication of the limits of functions is quite straightforward
        
        \[
            lim_{x \to c} [f(x) \cdot g(x)] \implies \lim_{x \to c} f(x) \cdot \lim_{x \to c} g(x)    
        \]

        The same exception applies when one of the limits is \uit{undefined}. This just makes
        the entire combined limit undefined

        \item \textbf{Division}
        
        Division is basically the same as the other basic operations except if the denominator is 0
        \[
            \lim_{x \to c} \frac{f(x)}{g(x)} \implies \frac{\lim_{x \to c} f(x)}{\lim_{x \to c} g(x)}   
        \]

        \item \textbf{Composite Functions}
        
        When working with composite functions, it's the same thing as taking
        the limit of the inner function, then evaluating the outer function normally
        \[
            \lim_{x \to c} f\bigg(g(x)\bigg) \implies f\bigg(\lim_{x \to c} g(x)\bigg)    
        \]

    \end{itemize}

    \subsection{Solving Limits}

        The first thing to always try to do when solving limits is \textbf{direct substitution}. If this
        is not possible (undefined limit), then algebraic manipulation (factoring) is the next step

        \[
            \begin{aligned}
                & lim_{x \to c} \frac{x^4 + 3x^3 - 10x^2}{x^2 - 2x} \\
                &= \lim_{x \to can} \frac{x^2(x^2 + 3x - 10)}{x(x - 2)} \\
                &= \lim_{x \to c} \frac{x^2(x+5)(x-2)}{x(x-2)}\\
                &= \lim_{x \to c} x^2(x+5)\\
            \end{aligned}    
        \]

        When encountering radicals, conjugates can be used.

        \[
            \begin{aligned}
                & \lim_{x \to c} \frac{x + 4}{\sqrt{3x + 13} - 1} \\
                &= \lim_{x \to c} \frac{x + 4}{\sqrt{3x + 13} - 1} \cdot \frac{\sqrt{3x + 13} + 1}{\sqrt{3x + 13} + 1} \\
                &= \lim_{x \to c} \frac{(x + 4)(\sqrt{3x + 13} + 1)}{3x + 12} \\
                &= \lim_{x \to c} \frac{(x + 4)(\sqrt{3x + 13} + 1)}{3(x + 4)} \\
                &= \lim_{x \to c} \frac{\sqrt{3x + 13} + 1}{3}
            \end{aligned}    
        \]

        When dealing with trigonometric equations, trig identities can be used
        (assuming direct substitution doesn't work)

        \[
            \begin{aligned}
                &\lim_{x \to c} \frac{\cot^2(x)}{1 - \sin(x)} \\
                &= \lim_{x \to c} \frac{\cos^2(x)}{(\sin^2(x))(1 - \sin(x))} \\
                &= \lim_{x \to c} \frac{1 - \sin^2(x)}{(\sin^2(x))(1 - \sin(x))} \\
                &= \lim_{x \to c} \frac{(1 + \sin(x))(1 - \sin(x))}{(\sin^2(x))(1 - \sin(x))} \\
                &= \lim_{x \to c} \frac{1 + \sin(x)}{\sin^2(x)}, \text{for x} \ne (2k + 1)\frac{\pi}{2}
            \end{aligned}    
        \]

        However, functions can not always be factored, so in that case they will just be undefined

        \[
            \begin{aligned}
                & \lim_{x \to 1} \frac{2x}{x^2 - 7x + 6} \\
                &= \lim_{x \to 1} \frac{2x}{(x -6)(x -1)} \\
                &= \frac{2}{0} \\
                &= \text{undef}
            \end{aligned}    
        \]

        \subsection{Continuity}

        

\end{document}